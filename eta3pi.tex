%%%%%%%%%%%%%%%%%%%%%%%%%%%%%%%%%%%%%%%%%%%%%
%%%%%                                   %%%%%
%%%%%       2014.8.29,   afternoon      %%%%%
%%%%%          1.1st version            %%%%%
%%%%%                                   %%%%%
%%%%%%%%%%%%%%%%%%%%%%%%%%%%%%%%%%%%%%%%%%%%%
\RequirePackage{lineno}
%\documentclass[prd,amsmath,amssymb,showpacs,superscriptaddress,nofootinbib,twocolumn]{revtex4}
\documentclass[prd,amsmath,amssymb,showpacs,superscriptaddress,nofootinbib]{revtex4}
\usepackage{graphicx}
\usepackage{epsfig}
\usepackage{dcolumn}
\usepackage{bm}
\usepackage{overpic}
\setlength{\parskip}{0\baselineskip}
\newcommand{\ra}{\rightarrow}
\newcommand{\BR}{{\cal B}}
\newcommand{\eff}{\varepsilon}
\newcommand{\LL}{\ell^+\ell^-}
\newcommand{\jpsi}{J/\psi}
\newcommand{\pio}{\pi^{0}}
\newcommand{\pip}{\pi^{+}}
\newcommand{\pim}{\pi^{-}}
\newcommand{\etap}{\eta^{\prime}}
\newcommand{\chisq}{\chi^{2}}
\renewcommand{\thefootnote}{\fnsymbol{footnote}}
%\pagewiselinenumbers
%\linenumbers
\begin{document}
\normalsize
\parskip=0pt plus 1pt minus 1pt

\title{\boldmath Matrix elements measurements for $\eta/\etap \ra 3\pi$  }

\author{M.~Ablikim$^{1}$, M.~N.~Achasov$^{8,a}$, X.~C.~Ai$^{1}$, O.~Albayrak$^{4}$, M.~Albrecht$^{3}$, D.~J.~Ambrose$^{42}$, A.~Amoroso$^{46A,46C}$, F.~F.~An$^{1}$, Q.~An$^{43}$, J.~Z.~Bai$^{1}$, R.~Baldini Ferroli$^{19A}$, Y.~Ban$^{29}$, D.~W.~Bennett$^{18}$, J.~V.~Bennett$^{4}$, M.~Bertani$^{19A}$, D.~Bettoni$^{20A}$, J.~M.~Bian$^{41}$, F.~Bianchi$^{46A,46C}$, E.~Boger$^{22,g}$, O.~Bondarenko$^{23}$, I.~Boyko$^{22}$, S.~Braun$^{38}$, R.~A.~Briere$^{4}$, H.~Cai$^{48}$, X.~Cai$^{1}$, O.~Cakir$^{37A}$, A.~Calcaterra$^{19A}$, G.~F.~Cao$^{1}$, S.~A.~Cetin$^{37B}$, J.~F.~Chang$^{1}$, G.~Chelkov$^{22,b}$, G.~Chen$^{1}$, H.~S.~Chen$^{1}$, J.~C.~Chen$^{1}$, M.~L.~Chen$^{1}$, S.~J.~Chen$^{27}$, X.~Chen$^{1}$, X.~R.~Chen$^{24}$, Y.~B.~Chen$^{1}$, H.~P.~Cheng$^{16}$, X.~K.~Chu$^{29}$, Y.~P.~Chu$^{1}$, G.~Cibinetto$^{20A}$, D.~Cronin-Hennessy$^{41}$, H.~L.~Dai$^{1}$, J.~P.~Dai$^{1}$, D.~Dedovich$^{22}$, Z.~Y.~Deng$^{1}$, A.~Denig$^{21}$, I.~Denysenko$^{22}$, M.~Destefanis$^{46A,46C}$, F.~De~Mori$^{46A,46C}$, Y.~Ding$^{25}$, C.~Dong$^{28}$, J.~Dong$^{1}$, L.~Y.~Dong$^{1}$, M.~Y.~Dong$^{1}$, S.~X.~Du$^{50}$, J.~Z.~Fan$^{36}$, J.~Fang$^{1}$, S.~S.~Fang$^{1}$, Y.~Fang$^{1}$, L.~Fava$^{46B,46C}$, F.~Feldbauer$^{21}$, G.~Felici$^{19A}$, C.~Q.~Feng$^{43}$, E.~Fioravanti$^{20A}$, C.~D.~Fu$^{1}$, Q.~Gao$^{1}$, Y.~Gao$^{36}$, I.~Garzia$^{20A}$, C.~Geng$^{43}$, K.~Goetzen$^{9}$, W.~X.~Gong$^{1}$, W.~Gradl$^{21}$, M.~Greco$^{46A,46C}$, M.~H.~Gu$^{1}$, Y.~T.~Gu$^{11}$, Y.~H.~Guan$^{1}$, L.~B.~Guo$^{26}$, T.~Guo$^{26}$, Y.~P.~Guo$^{21}$, Z.~Haddadi$^{23}$, S.~Han$^{48}$, Y.~L.~Han$^{1}$, F.~A.~Harris$^{40}$, K.~L.~He$^{1}$, M.~He$^{1}$, Z.~Y.~He$^{28}$, T.~Held$^{3}$, Y.~K.~Heng$^{1}$, Z.~L.~Hou$^{1}$, C.~Hu$^{26}$, H.~M.~Hu$^{1}$, J.~F.~Hu$^{46A}$, T.~Hu$^{1}$, G.~M.~Huang$^{5}$, G.~S.~Huang$^{43}$, H.~P.~Huang$^{48}$, J.~S.~Huang$^{14}$, X.~T.~Huang$^{31}$, Y.~Huang$^{27}$, T.~Hussain$^{45}$, Q.~Ji$^{1}$, Q.~P.~Ji$^{28}$, X.~B.~Ji$^{1}$, X.~L.~Ji$^{1}$, L.~L.~Jiang$^{1}$, L.~W.~Jiang$^{48}$, X.~S.~Jiang$^{1}$, J.~B.~Jiao$^{31}$, Z.~Jiao$^{16}$, D.~P.~Jin$^{1}$, S.~Jin$^{1}$, T.~Johansson$^{47}$, A.~Julin$^{41}$, N.~Kalantar-Nayestanaki$^{23}$, X.~L.~Kang$^{1}$, X.~S.~Kang$^{28}$, M.~Kavatsyuk$^{23}$, B.~C.~Ke$^{4}$, B.~Kloss$^{21}$, O.~B.~Kolcu$^{37B,c}$, B.~Kopf$^{3}$, M.~Kornicer$^{40}$, W.~Kuehn$^{38}$, A.~Kupsc$^{47}$, W.~Lai$^{1}$, J.~S.~Lange$^{38}$, M.~Lara$^{18}$, P.~Larin$^{13}$, M.~Leyhe$^{3}$, Cheng~Li$^{43}$, Cui~Li$^{43}$, D.~M.~Li$^{50}$, F.~Li$^{1}$, G.~Li$^{1}$, H.~B.~Li$^{1}$, J.~C.~Li$^{1}$, Jin~Li$^{30}$, K.~Li$^{12}$, K.~Li$^{31}$, Q.~J.~Li$^{1}$, T.~Li$^{31}$, W.~D.~Li$^{1}$, W.~G.~Li$^{1}$, X.~L.~Li$^{31}$, X.~N.~Li$^{1}$, X.~Q.~Li$^{28}$, Z.~B.~Li$^{35}$, H.~Liang$^{43}$, Y.~F.~Liang$^{33}$, Y.~T.~Liang$^{38}$, D.~X.~Lin$^{13}$, B.~J.~Liu$^{1}$, C.~L.~Liu$^{4}$, C.~X.~Liu$^{1}$, F.~H.~Liu$^{32}$, Fang~Liu$^{1}$, Feng~Liu$^{5}$, H.~B.~Liu$^{11}$, H.~H.~Liu$^{15}$, H.~M.~Liu$^{1}$, J.~Liu$^{1}$, J.~P.~Liu$^{48}$, K.~Liu$^{36}$, K.~Y.~Liu$^{25}$, Q.~Liu$^{39}$, S.~B.~Liu$^{43}$, X.~Liu$^{24}$, X.~X.~Liu$^{39}$, Y.~B.~Liu$^{28}$, Z.~A.~Liu$^{1}$, Zhiqiang~Liu$^{1}$, Zhiqing~Liu$^{21}$, H.~Loehner$^{23}$, X.~C.~Lou$^{1,d}$, H.~J.~Lu$^{16}$, J.~G.~Lu$^{1}$, R.~Q.~Lu$^{17}$, Y.~Lu$^{1}$, Y.~P.~Lu$^{1}$, C.~L.~Luo$^{26}$, M.~X.~Luo$^{49}$, T.~Luo$^{40}$, X.~L.~Luo$^{1}$, M.~Lv$^{1}$, X.~R.~Lyu$^{39}$, F.~C.~Ma$^{25}$, H.~L.~Ma$^{1}$, Q.~M.~Ma$^{1}$, S.~Ma$^{1}$, X.~Y.~Ma$^{1}$, F.~E.~Maas$^{13}$, M.~Maggiora$^{46A,46C}$, Q.~A.~Malik$^{45}$, Y.~J.~Mao$^{29}$, Z.~P.~Mao$^{1}$, S.~Marcello$^{46A,46C}$, J.~G.~Messchendorp$^{23}$, J.~Min$^{1}$, T.~J.~Min$^{1}$, R.~E.~Mitchell$^{18}$, X.~H.~Mo$^{1}$, Y.~J.~Mo$^{5}$, H.~Moeini$^{23}$, C.~Morales Morales$^{13}$, K.~Moriya$^{18}$, N.~Yu.~Muchnoi$^{8,a}$, H.~Muramatsu$^{41}$, Y.~Nefedov$^{22}$, F.~Nerling$^{13}$, I.~B.~Nikolaev$^{8,a}$, Z.~Ning$^{1}$, S.~Nisar$^{7}$, X.~Y.~Niu$^{1}$, S.~L.~Olsen$^{30}$, Q.~Ouyang$^{1}$, S.~Pacetti$^{19B}$, P.~Patteri$^{19A}$, M.~Pelizaeus$^{3}$, H.~P.~Peng$^{43}$, K.~Peters$^{9}$, J.~L.~Ping$^{26}$, R.~G.~Ping$^{1}$, R.~Poling$^{41}$, Y.~N.~Pu$^{17}$, M.~Qi$^{27}$, S.~Qian$^{1}$, C.~F.~Qiao$^{39}$, L.~Q.~Qin$^{31}$, N.~Qin$^{48}$, Y.~Qin$^{29}$, Z.~H.~Qin$^{1}$, J.~F.~Qiu$^{1}$, K.~H.~Rashid$^{45}$, C.~F.~Redmer$^{21}$, H.~L.~Ren$^{17}$, M.~Ripka$^{21}$, G.~Rong$^{1}$, X.~D.~Ruan$^{11}$, V.~Santoro$^{20A}$, A.~Sarantsev$^{22,e}$, M.~Savri��$^{20B}$, K.~Schoenning$^{47}$, S.~Schumann$^{21}$, W.~Shan$^{29}$, M.~Shao$^{43}$, C.~P.~Shen$^{2}$, X.~Y.~Shen$^{1}$, H.~Y.~Sheng$^{1}$, M.~R.~Shepherd$^{18}$, W.~M.~Song$^{1}$, S.~Spataro$^{46A,46C}$, B.~Spruck$^{38}$, S.~Stefano$^{46A,46C}$, G.~X.~Sun$^{1}$, J.~F.~Sun$^{14}$, S.~S.~Sun$^{1}$, Y.~J.~Sun$^{43}$, Y.~Z.~Sun$^{1}$, Z.~J.~Sun$^{1}$, Z.~T.~Sun$^{43}$, C.~J.~Tang$^{33}$, X.~Tang$^{1}$, I.~Tapan$^{37C}$, E.~H.~Thorndike$^{42}$, M.~Tiemens$^{23}$, D.~Toth$^{41}$, M.~Ullrich$^{38}$, I.~Uman$^{37B}$, G.~S.~Varner$^{40}$, B.~Wang$^{28}$, B.~L.~Wang$^{39}$, D.~Wang$^{29}$, D.~Y.~Wang$^{29}$, K.~Wang$^{1}$, L.~L.~Wang$^{1}$, L.~S.~Wang$^{1}$, M.~Wang$^{31}$, P.~Wang$^{1}$, P.~L.~Wang$^{1}$, Q.~J.~Wang$^{1}$, S.~G.~Wang$^{29}$, W.~Wang$^{1}$, X.~F.~Wang$^{36}$, Y.~D.~Wang$^{19A}$, Y.~F.~Wang$^{1}$, Y.~Q.~Wang$^{21}$, Z.~Wang$^{1}$, Z.~G.~Wang$^{1}$, Z.~H.~Wang$^{43}$, Z.~Y.~Wang$^{1}$, D.~H.~Wei$^{10}$, J.~B.~Wei$^{29}$, P.~Weidenkaff$^{21}$, S.~P.~Wen$^{1}$, M.~Werner$^{38}$, U.~Wiedner$^{3}$, M.~Wolke$^{47}$, L.~H.~Wu$^{1}$, N.~Wu$^{1}$, Z.~Wu$^{1}$, L.~G.~Xia$^{36}$, Y.~Xia$^{17}$, D.~Xiao$^{1}$, Z.~J.~Xiao$^{26}$, Y.~G.~Xie$^{1}$, Q.~L.~Xiu$^{1}$, G.~F.~Xu$^{1}$, L.~Xu$^{1}$, Q.~J.~Xu$^{12}$, Q.~N.~Xu$^{39}$, X.~P.~Xu$^{34}$, Z.~Xue$^{1}$, L.~Yan$^{43}$, W.~B.~Yan$^{43}$, W.~C.~Yan$^{43}$, Y.~H.~Yan$^{17}$, H.~X.~Yang$^{1}$, L.~Yang$^{48}$, Y.~Yang$^{5}$, Y.~X.~Yang$^{10}$, H.~Ye$^{1}$, M.~Ye$^{1}$, M.~H.~Ye$^{6}$, B.~X.~Yu$^{1}$, C.~X.~Yu$^{28}$, H.~W.~Yu$^{29}$, J.~S.~Yu$^{24}$, C.~Z.~Yuan$^{1}$, W.~L.~Yuan$^{27}$, Y.~Yuan$^{1}$, A.~Yuncu$^{37B,f}$, A.~A.~Zafar$^{45}$, A.~Zallo$^{19A}$, S.~L.~Zang$^{27}$, Y.~Zeng$^{17}$, B.~X.~Zhang$^{1}$, B.~Y.~Zhang$^{1}$, C.~Zhang$^{27}$, C.~C.~Zhang$^{1}$, D.~H.~Zhang$^{1}$, H.~H.~Zhang$^{35}$, H.~Y.~Zhang$^{1}$, J.~J.~Zhang$^{1}$, J.~Q.~Zhang$^{1}$, J.~W.~Zhang$^{1}$, J.~Y.~Zhang$^{1}$, J.~Z.~Zhang$^{1}$, S.~H.~Zhang$^{1}$, X.~J.~Zhang$^{1}$, X.~Y.~Zhang$^{31}$, Y.~Zhang$^{1}$, Y.~H.~Zhang$^{1}$, Z.~H.~Zhang$^{5}$, Z.~P.~Zhang$^{43}$, Z.~Y.~Zhang$^{48}$, J.~W.~Zhao$^{1}$, Lei~Zhao$^{43}$, Ling~Zhao$^{1}$, M.~G.~Zhao$^{28}$, Q.~Zhao$^{1}$, Q.~W.~Zhao$^{1}$, S.~J.~Zhao$^{50}$, T.~C.~Zhao$^{1}$, Y.~B.~Zhao$^{1}$, Z.~G.~Zhao$^{43}$, A.~Zhemchugov$^{22,g}$, B.~Zheng$^{44}$, J.~P.~Zheng$^{1}$, Y.~H.~Zheng$^{39}$, B.~Zhong$^{26}$, L.~Zhou$^{1}$, Li~Zhou$^{28}$, X.~Zhou$^{48}$, X.~R.~Zhou$^{43}$, X.~Y.~Zhou$^{1}$, K.~Zhu$^{1}$, K.~J.~Zhu$^{1}$, X.~L.~Zhu$^{36}$, Y.~C.~Zhu$^{43}$, Y.~S.~Zhu$^{1}$, Z.~A.~Zhu$^{1}$, J.~Zhuang$^{1}$, B.~S.~Zou$^{1}$, J.~H.~Zou$^{1}$
\\
\vspace{0.2cm}
(BESIII Collaboration)\\
\vspace{0.2cm} {\it
$^{1}$ Institute of High Energy Physics, Beijing 100049, People's Republic of China\\
$^{2}$ Beihang University, Beijing 100191, People's Republic of China\\
$^{3}$ Bochum Ruhr-University, D-44780 Bochum, Germany\\
$^{4}$ Carnegie Mellon University, Pittsburgh, Pennsylvania 15213, USA\\
$^{5}$ Central China Normal University, Wuhan 430079, People's Republic of China\\
$^{6}$ China Center of Advanced Science and Technology, Beijing 100190, People's Republic of China\\
$^{7}$ COMSATS Institute of Information Technology, Lahore, Defence Road, Off Raiwind Road, 54000 Lahore, Pakistan\\
$^{8}$ G.I. Budker Institute of Nuclear Physics SB RAS (BINP), Novosibirsk 630090, Russia\\
$^{9}$ GSI Helmholtzcentre for Heavy Ion Research GmbH, D-64291 Darmstadt, Germany\\
$^{10}$ Guangxi Normal University, Guilin 541004, People's Republic of China\\
$^{11}$ GuangXi University, Nanning 530004, People's Republic of China\\
$^{12}$ Hangzhou Normal University, Hangzhou 310036, People's Republic of China\\
$^{13}$ Helmholtz Institute Mainz, Johann-Joachim-Becher-Weg 45, D-55099 Mainz, Germany\\
$^{14}$ Henan Normal University, Xinxiang 453007, People's Republic of China\\
$^{15}$ Henan University of Science and Technology, Luoyang 471003, People's Republic of China\\
$^{16}$ Huangshan College, Huangshan 245000, People's Republic of China\\
$^{17}$ Hunan University, Changsha 410082, People's Republic of China\\
$^{18}$ Indiana University, Bloomington, Indiana 47405, USA\\
$^{19}$ (A)INFN Laboratori Nazionali di Frascati, I-00044, Frascati, Italy; (B)INFN and University of Perugia, I-06100, Perugia, Italy\\
$^{20}$ (A)INFN Sezione di Ferrara, I-44122, Ferrara, Italy; (B)University of Ferrara, I-44122, Ferrara, Italy\\
$^{21}$ Johannes Gutenberg University of Mainz, Johann-Joachim-Becher-Weg 45, D-55099 Mainz, Germany\\
$^{22}$ Joint Institute for Nuclear Research, 141980 Dubna, Moscow region, Russia\\
$^{23}$ KVI-CART, University of Groningen, NL-9747 AA Groningen, The Netherlands\\
$^{24}$ Lanzhou University, Lanzhou 730000, People's Republic of China\\
$^{25}$ Liaoning University, Shenyang 110036, People's Republic of China\\
$^{26}$ Nanjing Normal University, Nanjing 210023, People's Republic of China\\
$^{27}$ Nanjing University, Nanjing 210093, People's Republic of China\\
$^{28}$ Nankai University, Tianjin 300071, People's Republic of China\\
$^{29}$ Peking University, Beijing 100871, People's Republic of China\\
$^{30}$ Seoul National University, Seoul, 151-747 Korea\\
$^{31}$ Shandong University, Jinan 250100, People's Republic of China\\
$^{32}$ Shanxi University, Taiyuan 030006, People's Republic of China\\
$^{33}$ Sichuan University, Chengdu 610064, People's Republic of China\\
$^{34}$ Soochow University, Suzhou 215006, People's Republic of China\\
$^{35}$ Sun Yat-Sen University, Guangzhou 510275, People's Republic of China\\
$^{36}$ Tsinghua University, Beijing 100084, People's Republic of China\\
$^{37}$ (A)Ankara University, Dogol Caddesi, 06100 Tandogan, Ankara, Turkey; (B)Dogus University, 34722 Istanbul, Turkey; (C)Uludag University, 16059 Bursa, Turkey\\
$^{38}$ Universitaet Giessen, D-35392 Giessen, Germany\\
$^{39}$ University of Chinese Academy of Sciences, Beijing 100049, People's Republic of China\\
$^{40}$ University of Hawaii, Honolulu, Hawaii 96822, USA\\
$^{41}$ University of Minnesota, Minneapolis, Minnesota 55455, USA\\
$^{42}$ University of Rochester, Rochester, New York 14627, USA\\
$^{43}$ University of Science and Technology of China, Hefei 230026, People's Republic of China\\
$^{44}$ University of South China, Hengyang 421001, People's Republic of China\\
$^{45}$ University of the Punjab, Lahore-54590, Pakistan\\
$^{46}$ (A)University of Turin, I-10125, Turin, Italy; (B)University of Eastern Piedmont, I-15121, Alessandria, Italy; (C)INFN, I-10125, Turin, Italy\\
$^{47}$ Uppsala University, Box 516, SE-75120 Uppsala, Sweden\\
$^{48}$ Wuhan University, Wuhan 430072, People's Republic of China\\
$^{49}$ Zhejiang University, Hangzhou 310027, People's Republic of China\\
$^{50}$ Zhengzhou University, Zhengzhou 450001, People's Republic of China\\
\vspace{0.2cm}
$^{a}$ Also at the Novosibirsk State University, Novosibirsk, 630090, Russia\\
$^{b}$ Also at the Moscow Institute of Physics and Technology, Moscow 141700, Russia and at the Functional Electronics Laboratory, Tomsk State University, Tomsk, 634050, Russia \\
$^{c}$ Currently at Istanbul Arel University, Kucukcekmece, Istanbul, Turkey\\
$^{d}$ Also at University of Texas at Dallas, Richardson, Texas 75083, USA\\
$^{e}$ Also at the PNPI, Gatchina 188300, Russia\\
$^{f}$ Also at Bogazici University, 34342 Istanbul, Turkey\\
$^{g}$ Also at the Moscow Institute of Physics and Technology, Moscow 141700, Russia\\
}
}

\date{\today}

\linenumbers

\begin{abstract}
Based on 1.3 billion $\jpsi$ data collected with the BESIII detector, we present the measurements of matrix elements for the $\eta \ra 3\pi$ decays. Using an un-binned maximum likelihood method to fit the Dalitz plot density distribution, we obtain the parameters precisely, which are in reasonable agreement with previous results. We also performed the slope parameter measurement for $\etap \ra \pio\pio\pio$ with the largest $\etap$ samples.
\end{abstract}

%\pacs{13.25.Gv, 13.66.Bc, 14.40.Pq, 14.40.Rt}
\maketitle

\section{INTRODUCTION}

As is known to all, the decays $\eta/\etap\ra3\pi$ are accompanied with iso-spin violation. According to the Sutherland theorem\cite{ref:chpt}, electromagnetic interaction contributes very small to those processes and they are induced dominantly by the strong interaction via the u-d quark mass difference. Therefore, it's an ideal laboratory for testing chiral perturbation theory, CHPT. Those decays also provide validation of models for $\pi-\pi$ final state interaction.

For the three body decay of a spin-0 particle, there are only two free variables. For the decays $\eta\ra\pip\pim\pio$, they can be chosen as
\begin{equation}
    X=\frac{\sqrt{3}}{Q}(T_{\pip}-T_{\pim}),
    Y=\frac{m_{\pip}+m_{\pim}+m_{\pio}}{m_\pi}\frac{T_{\pio}}{Q},
\end{equation}
where $T_{\pi}$ denote the kinetic energies in the $\eta$ rest frame and $Q=m_{\eta}-m_{\pip}-m_{\pim}-m_{\pio}$. For the charged channel, the decay amplitude square can be expanded as the polynomial of X and Y

\begin{equation}\label{eq:etacha_amp}
	|A(X,Y)|^{2} = N(1+aY+bY^2+cX+dX^2+eXY+fY^{3}+\ldots),
\end{equation}
where a, b, c, d, e and f are real parameters and N is the normalized factor. The non-zero odd power of X (c and e) imply the charge conjugation violation.

Due to the symmetry of three neutral pions, it is convenient to use one fully symmetrized variable for the decays $\eta/\etap\ra\pio\pio\pio$
\begin{equation}
    Z=X^2+Y^2=\frac{2}{3}\sum^{3}_{i=1}(\frac{3T_{i}}{Q}-1)^{2}.
\end{equation}
Then parametrization of Dalitz plot reads
\begin{equation}\label{eq:etaneu_amp}
    |A(Z)|^2=N(1+2\alpha Z),
\end{equation}
where $\alpha$ is the slope parameter of the Dalitz plot while N is the normalized factor.

Many collaborations, such as KLOE\cite{ref:kloe}\cite{ref:etaneukloe}, CBRA\cite{ref:cbar}\cite{ref:etaneucbar}, ASPK\cite{ref:aspk} etc., have measured the matrix elements of the decays $\eta\ra{}3\pi$. For the decay $\etap\ra{}3\pio$, GAMS2\cite{ref:etapneugam2} and GAM\cite{ref:etapneugam} have measured the related Dalitz plot parameter and the results showed large discrepancy.

In this paper, the matrix elements measurements for the $\eta \ra 3\pi$ decays are performed precisely based on
about 1.3 billion $\jpsi$ events accumulated by BESIII at BEPCII\@. With a new level of precision, we also present the results of the slope parameter for $\etap\ra\pio\pio\pio$\.

\section{DETECTOR AND MONTE CARLO SIMULATION}

BEPCII\cite{bepc2_design} is a double-ring $e^{+}e^{-}$ collider designed to provide a peak luminosity of $10^{33}$ cm$^{-2}s^{-1}$ at the c.m.\ energy of $3.770$ GeV. The BESIII\cite{Ablikim2010345} detector, with a geometrical acceptance of 93\% of 4$\pi$ stereo angle, is operating in 1 T magnetic field provided by a superconducting solenoid magnet. It is composed of a helium-based drift chamber (MDC), a plastic scintillator Time-Of-Flight (TOF) system, a CsI (Tl) electromagnetic Calorimeter (EMC) and a multi-layer resistive plate counter (RPC) system (MUC). The charged-particle momentum resolution at 1 $GeV/c^2$ is 0.5\%, and the $dE/dx$ resolution is better than 6\%. The spatial resolution of MDC is better than 130 $\mu$m. The time resolution of TOF is 80 ps in the barrel and 110 ps in the endcaps. The energy resolution for EMC measurement is better than 2.5\% in the barrel and 5\% in the endcaps and position resolution is better than 6 mm for 1 GeV/c electrons and photons. The spatial resolution in MUC is better than 2 cm.

Monte Carlo (MC) simulations are used to estimate backgrounds and determine the detection efficiencies. The GEANT4-based simulation software BOOST\cite{ref:boost} includes the geometric and material description of the BESIII detectors, detector response, and digitization models, as well as the tracking of the detector running conditions and performance. The production of the $\jpsi$ resonance is simulated by the MC event generator KKMC\cite{ref:kkmc}\cite{ref:kkmc2}, while the decays are generated by EVTGEN\cite{ref:evtgen} for known decay modes with branching fractions being set to the PDG\cite{ref:pdg} world average values, and by LUNDCHARM\cite{ref:lundcharm} for the remaining unknown decays. The analysis is performed in the framework of the BESIII offline software system (BOSS)\cite{ref:boss} which takes care of the detector calibration, event reconstruction, and data storage.

\section{EVENT SELECTION}

The charged-particle tracks are reconstructed from hits in MDC within the polar angle range $|\cos\theta| < 0.93$. Tracks that extrapolate to be within 10 cm of the interaction point in the beam direction and 1 cm in the plane perpendicular to the beam are selected. Photon candidates are reconstructed by isolated showers in EMC and required to have at least 25 MeV of energy in barrel ($|$cos$\theta|$ $<$ 0.80) and 50 MeV in end-caps (0.86 $<$ $|$cos$\theta|$ $<$ 0.92). To eliminate showers associated with charged particles, the angle between the cluster and the nearest charged track must be more than 10 degrees. EMC cluster timing requirements are also used to suppress electronic noise and energy deposits unrelated to the event.

For $\jpsi\ra\gamma\eta$ with $\eta\ra\pip\pim\pio$, the candidate events are required to have two opposite charged tracks and at least three photon candidates. The photon with the maximum energy is regarded as from $\jpsi$. A six-constraint (6C) kinematic fit imposing energy-momentum conservation and $\pio$ and $\eta$ mass constraints is performed under the $\gamma\gamma\gamma\pip\pim$ hypothesis and the $\chi^{2}_{6C}$ is required to be less than 80. If there is more than three photons in an event, the combination with the smallest $\chi^{2}_{6C}$ is retained and the probability is required to be larger than that for the $\gamma\gamma\gamma\gamma\pip\pim$ hypothesis. To suppress two photons final states, the probability for 6C kinematic fit must be larger than that for the $\gamma\gamma\pip\pim$ hypothesis.

For $\jpsi\ra\gamma\eta/\etap$ with $\eta/\etap\ra\pio\pio\pio$, candidate events must have no good charged track and at least seven photons. One-constraint (1C) kinematic fits are performed on the $\pio$ candidates reconstructed from photon pairs with the invariant mass of the two photons being constrained to the nominal $\pio$ mass, and $\chi^{2}_{1C}(\gamma\gamma) < 25$ is required.
To remove wrong photon combinations, the $\pio$ decay angle, defined as the polar angle of a photon in the
$\pio$ rest frame, is required to satisfy $|\cos\theta_{decay}| < 0.95$.  Then a seven-constraint (7C) kinematic fit (three $\pio$ masses are also constrained) is performed under the hypothesis of $\jpsi\ra\gamma\pio\pio\pio$ and $\chi^{2}_{7C} < 70$ is required. For events with more than three $\pio$ candidates, the combination with the smallest $\chi^{2}_{7C}$ is retained.

In the decay $\etap\ra\pio\pio\pio$, events with $|M_{\gamma\pio}-m_{\omega}| > 0.05$ GeV/$c^{2}$ are rejected to suppress the background from $\jpsi\ra\omega\pio\pio$. The requirements $\chisq(\gamma\pio\pio\pio) < \chisq(\gamma\eta\pio\pio)$ and $|M_{\gamma\gamma} - m_{\eta}| > 0.03$ GeV/$c^{2}$ are used to reject the peaking backgrounds from $\etap\ra\eta\pio\pio$ as many as possible.

The potential background channels listed in PDG\cite{ref:pdg} in the selected event samples are studied by inclusive MC simulations. The total background contamination is estimated to be only 0.1\% and 0.07\% for $\eta\ra\pip\pim\pio$ and $\eta\ra\pio\pio\pio$, respectively. But there are some non-peaking backgrounds in the decay $\etap\ra\pio\pio\pio$ which can be described by the normalized $\etap$ mass sideband events. The peaking backgrounds $\etap\ra\eta\pio\pio$ can be studied by MC simulation based on the results measured in GAM4\cite{ref:etapipi}.

\section{MEASUMENT OF THE MATRIX ELEMENTS FOR DALITZ PLOT OF $\eta\ra\pip\pim\pio$}

After the above event selection, 79625 events are survived in total. Fig.~\ref{fig:etacha_dalXY_datamc} (a) shows the experimental form of the Dalitz plot for the decay $\eta\ra\pip\pim\pio$ in terms of the variables X and Y.
The corresponding projections on X and Y are shown in Fig.~\ref{fig:etacha_dalXY_datamc} (b) and Fig.~\ref{fig:etacha_dalXY_datamc} (c) respectively, where the black dots represent data while the red solid histograms are from a MC signal sample with $\eta\ra\pip\pim\pio$ events produced with phase space. The resolution in the variables X and Y over the entire kinematical region are $\sigma_{X} = 0.024$ and $\sigma_{Y} = 0.021$ according to MC simulation.

\begin{figure}[!htbp]
    \begin{center}
        \begin{overpic}[width=0.3\textwidth]{figures/etacha/etacha_dalXY_data.eps}
            \put(25,58){\bf (a)}
        \end{overpic}
        \begin{overpic}[width=0.3\textwidth]{figures/etacha/etacha_dalX_datamc.eps}
            \put(25,58){\bf (b)}
        \end{overpic}
        \begin{overpic}[width=0.3\textwidth]{figures/etacha/etacha_dalY_datamc.eps}
            \put(25,58){\bf (c)}
        \end{overpic}
    \end{center}
    \caption{\label{fig:etacha_dalXY_datamc} (a) The experimental form of the Dalitz diagram for the decay $\eta\ra\pip\pim\pio$ in terms of the variables X and Y. The corresponding projections on variable X and Y are shown in (b) and (c) respectively, where the black dots are from data and the histograms are from a MC signal sample with $\eta\ra\pip\pim\pio$ events produced with phase space.}
\end{figure}

In order to describe the event density distribution on X and Y, efficiency correction is performed and the probability density function (p.d.f.) p(X,Y) is used as
\begin{equation}
	p(X,Y) =  \frac{|A(X,Y)|^{2}\varepsilon(X,Y)}{\int_{DP}{|A(X,Y)|^{2}\varepsilon(X,Y)dXdY}},
\end{equation}
where the detection efficiency $\varepsilon(X,Y)$ is determined according to KLOE's results on the analysis of $\eta \ra \pip\pim\pio$ and second-order polynomial functions are used to parameterize it, while $|A(X,Y)|^{2}$ is described as Eq.~(\ref{eq:etacha_amp}) and the integral limit DP denotes the kinematic limit of the Dalitz plot. The p.d.f. free parameters are optimized with a maximum likelihood fit, where the log-likelihood function is described as
\begin{equation}
	-\ln {\cal L}(X,Y) = -\sum_{i=1}^{N_{event}}\ln{}{p_{i}(X,Y)},
\end{equation}
where $N_{event}$ is the number of events in the sample to parameterize. The fit procedure has been verified with MC by checking the input and output values of the Dalitz plot parameters and all the biases are found to be small compared to the statistical errors.

To evaluate the goodness of fit, a $\chisq$ variable on the binned Dalitz plot can be defined as
\begin{eqnarray}
     \chisq & = & \sum_{i=1}^{N_{bins}}\frac{(n^{rec}_{i}-n^{fit}_{i})^{2}}{n^{fit}_{i}},
\end{eqnarray}
where $N_{bins}$ is the number of the bins, $n_{i}^{rec}$ is the number of events observed in the $i^{th}$ bin, and $n_{i}^{fit}$ is the number predicted from the fitted p.d.f.
%The $\chisq$ follows $\chi^{2}$ distribution whose freedom is ($N_{bins}$-L-1), where L is the number of free parameters in p.d.f.

Fitting to data gives the following values for the matrix element parameters and for the correlation matrix with $\chisq/NDF=1400/1153$ (the NDF is the number of degrees of freedom)

\begin{equation}
\begin{matrix}
a = -1.128\pm0.016 \\
b = 0.154\pm0.017 \\
10^{2}c = 0.058\pm0.850 \\
d = 0.085\pm0.016 \\
e = 0.017\pm0.019 \\
f = 0.173\pm0.031
\end{matrix}
\begin{pmatrix}
1.000 &  -0.242 & -0.004 & -0.407 & 0.003 & -0.772 \\
      &   1.000 & -0.001 &  0.324 & 0.013 & -0.309 \\
      &         &  1.000 &  0.002 &-0.591 &  0.005 \\
      &         &        & 1.000  & 0.013 &  0.104 \\
      &         &        &        & 1.000 & -0.008 \\
      &         &        &        &       &  1.000 \\
\end{pmatrix}
\end{equation}

The errors are statistical only. The results are illustrated in Fig.~\ref{fig:etacha_dalXY_fit}, where we show the comparison of data (dots with error bars) and MC weighted with fitted coefficients (histogram) as a function of X and Y variables. As expected from the charge conjugation symmetry, the fitted values of parameter c and e are consistent with zero within 1 $\sigma$\@.

\begin{figure}[!htbp]
	\begin{center}
		\begin{overpic}[width=0.45\textwidth]{figures/etacha/etacha_dalX_fit.eps}
			\put(25,58){\bf (a)}
		\end{overpic}
		\begin{overpic}[width=0.45\textwidth]{figures/etacha/etacha_dalY_fit.eps}
			\put(25,58){\bf (b)}
		\end{overpic}
	\end{center}
    \caption{\label{fig:etacha_dalXY_fit} The distributions of the Dalitz plot variables X (a) and Y (b), the dots with error bars are data and histograms are the fitted functions.}
\end{figure}

We've also done the fit while fixing the parameter c and e to zero just as other experiments did and all the other parameters are almost the same. This fitting yields the following results with $\chisq/NDF=1402/1151$

\begin{equation}
\begin{matrix}
a = -1.122\pm0.016 \\
b = 0.154\pm0.017 \\
d = 0.085\pm0.016  \\
f = 0.173\pm0.031 \\
\end{matrix}
\begin{pmatrix}
1.000 &  -0.242 & -0.407 & -0.772\\
      &   1.000 &  0.324 & -0.309\\
      &         &  1.000 & 0.104 \\
      &         &        & 1.000
\end{pmatrix}
\end{equation}

\section{MEASUMENT OF THE MATRIX ELEMENT FOR DALITZ PLOT OF $\eta\ra\pio\pio\pio$}
To obtain Z distribution correctly, we use the information after 8C (four momentum constraint of $\jpsi$, the nominal mass of pions and $\eta$) kinematic fit.
Fig.~\ref{fig:etaneu_dalZ_datamc} (a) shows the Z distribution for data (black dots) and a MC signal sample with $\eta \ra \pio\pio\pio$ events produced with phase space (the red histogram). 
For PHSP MC distribution, Z distribution is flat from $Z = 0$ to $Z \approx 0.76$ and then falls to zero at $Z = 1$ due to the kinematic boundaries and we'll choose the fit range as $(0, 0.7)$ and 27672 events are selected.

\begin{figure}[!htbp]
    \begin{center}
        %\begin{overpic}[width=0.4\textwidth]{figures/etaneu/etaneu_dalZ_mc.eps}
        \begin{overpic}[width=0.4\textwidth]{figures/etaneu/etaneu_dalZ_mcdt.eps}
            \put(85,60){\bf (a)}
        \end{overpic}
        \begin{overpic}[width=0.4\textwidth]{figures/etaneu/etaneu_dalfit.eps}
            \put(85,60){\bf (b)}
        \end{overpic}
    \end{center}
    \caption{\label{fig:etaneu_dalZ_datamc} (a) The Dalitz plot in terms of Z for phase space MC sample of $\eta\ra\pio\pio\pio$ and data, 
	(b) Fit to the experimental form of the Dalitz diagram.}
\end{figure}

To describe the event density distribution on Z, resolution and efficiency corrections are performed and the p.d.f.\ described as \begin{equation}
    p(Z) = |A(Z)|^2\otimes{}R(Z)\times\varepsilon(Z),
\end{equation}
where $|A(Z)|^{2}$ is described in Eq.~(\ref{eq:etaneu_amp}), the resolution $R(Z)$ and the efficiency $\varepsilon(Z)$, which is parameterized with $1^{st}$ order Chebychev Polynomial, are determined based on a phase space MC signal sample. The free parameter $\alpha$ is optimized with an un-binned maximum likelihood fit that minimizes a log-Likelihood function built as follows
\begin{equation}
	-\ln{\cal L}(Z) = -\sum_{i=1}^{N_{event}}\ln{}p_{i}(Z),
\end{equation}
where $N_{event}$ is the number of events within our fit range. Fitting to the data gives a result $\alpha = -0.055 \pm 0.014$ where the error is statistical only with $\chi^{2}/NDF = 11.4/12$. The result is illustrated in Fig.~\ref{fig:etaneu_dalZ_datamc} (b).
%our fit procedure has been verified with MC
%with checking the input and output value.

\section{MEASUMENT OF THE MATRIX ELEMENT FOR DALITZ PLOT OF $\etap\ra\pio\pio\pio$}

Due to the peaking background from $\etap\ra\eta\pio\pio$ and non-peaking backgrounds, background subtraction in the fit should be performed and 7C(four momentum of $\jpsi$ and masses of pions) information is used for measurement. The log-likelihood function is built as
\begin{equation}\label{eq:likelyhood}
	-\ln{\cal L}(Z) = -(\sum_{i=1}^{N_{data}}\ln{}p_{i}(Z) - a\sum_{j=1}^{N_{pkg}}\ln{}p_{j}(Z)-b\sum_{k=1}^{N_{non-pkg}}\ln{}p_{k}(Z)),
\end{equation}
where $a\sum_{j=1}^{N_{pkg}}\ln{}p_{j}(Z)$ is for $\eta'\ra\eta\pio\pio$ peaking background which is estimated from the MC simulation based on the matrix elements result in Ref.~\cite{ref:etapipi}. $N_{pkg}$ is the number of events for MC sample after the final selection and a is the normalized factor based on the branching ratio. $b\sum_{k=1}^{N_{non-pkg}}\ln{}p_{k}(Z)$ is for other backgrounds which is described by $\etap$ sideband events and b is the normalized factor set to be 1 due to nearly the same background events number in the signal and sideband regions. The sideband regions are chosen as $(0.758, 0.793)$ GeV/$c^{2}$ and $(1.008, 1.043)$ GeV/$c^{2}$.

Fig.~\ref{fig:etapneu_dalZ_datamc} (a) shows the Z distribution for the pure phase space MC sample and data with subtraction of backgrounds. According to the flat part of PHSP MC distribution, we choose the fit range as $[0, 0.45]$. About 2000 events are included in the fit range after the subtraction. As showed in Fig.~\ref{fig:etapneu_dalZ_datamc} (b), dots with errors bars are for data with subtraction of backgrounds and the blue line represents the fitted result. The fitting gives $\alpha = -0.687 \pm 0.049$ where the error is statistical only and the goodness of fit is $\chi^{2}/NDF = 9.9/7$.

\begin{figure}[htbp]
    \begin{center}
        %\begin{overpic}[width=0.4\textwidth]{figures/etapneu/etapneu_dalZ_mc.eps}
        \begin{overpic}[width=0.4\textwidth]{figures/etapneu/etapneu_dalZ_mcdt.eps}
            \put(85,60){\bf (a)}
        \end{overpic}
        \begin{overpic}[width=0.4\textwidth]{figures/etapneu/etapneu_dalfit.eps}
            \put(85,60){\bf (b)}
        \end{overpic}
    \end{center}
    \caption{\label{fig:etapneu_dalZ_datamc} (a) The Dalitz plot in terms of Z for phase space MC sample of $\etap\ra\pio\pio\pio$ adn data, 
	(b) Fit to the experimental form of the Dalitz diagram.}
\end{figure}

\section{SYSTEMATIC UNCERTAINTIES}
%The sources of the systematic uncertainties for $\eta\ra\pip\pim\pio$, $\eta\ra\pio\pio\pio$ and $\etap\ra\pio\pio\pio$ are listed in Table~\ref{tab:etacha_syserr}, ~\ref{tab:etaneu_syserr} and ~\ref{tab:etapneu_syserr} respectively.

The systematic errors in the measurement of the Dalitz plot matrix element for $\eta\ra\pip\pim\pio$, $\eta\ra\pio\pio\pio$ and $\etap\ra\pio\pio\pio$ are summarized in Table~\ref{tab:etacha_syserr}. The uncertainties from the fitting method (i.e.\ fitting bias) are estimated by the input-output checks. The uncertainties associated with efficiency mainly come from the parametrization of efficiency and the difference between data and MC simulation. For the uncertainties due to the efficiency parametrization, we change the global polynomial fit to average efficiencies of local bins. The differences from the nominal values are taken as the systematic error. The tracking efficiency correction functions for $\pip$ and $\pim$ are obtained by using the control sample $\jpsi\ra\pip\pim p\bar{p}$, where the transverse momentum region of the pion has covered the region of signal pion transverse momentum. The difference on the fitted parameters are taken as the systematic error. The $\pio$ detection efficiency correction function is obtained by using the control sample $\jpsi\ra\pip\pim\pio$ and the difference on the fitted parameters are taken as the systematic error. 
%The fitted results are also compared using a 5C instead of a 6C kinematic fit for $\eta\ra\pip\pim\pio$, 7C instead of 8c for $\eta\ra\pio\pio\pio$, and the corresponding differences are taken as the systematic errors due to the $\eta$ constrain uncertainty.
The fitted results are also compared between two different kinematic fits (with and without the $\eta$ mass constraint) for $\eta\ra\pip\pim\pio$ and $\eta\ra\pio\pio\pio$ to estimate the systematic errors due to the $\eta$ constrain uncertainty.
For $\eta\ra\pip\pim\pio$, since the unknown correlation between X and Y variables, we can only estimate the uncertainty from the resolution of X and Y ignoring the correlation and the resolution functions are numerically convoluted with the p.d.f.\ while fitting to the data. The variations from the nominal values are taken as the systematic uncertainty.
For $\eta\ra\pio\pio\pio$ and $\etap\ra\pio\pio\pio$, the differences on the fitted parameters due to different fit range are also included in the systematic errors. For $\etap\ra\pio\pio\pio$, the backgrounds estimation is one of the sources. The uncertainties from peaking background are mainly from the errors of branching fractions and the shape which is estimated by changing the input matrix values for $\etap\ra\eta\pio\pio$ to other values listed in Ref.~\cite{ref:etapipi} in MC simulation. The uncertainty from non-peaking backgrounds is estimated by changing the sideband regions. Assuming all the sources are independent and adding them in quadrature, we can get the total systematic errors for the matrix elements shown in the last row in Table~\ref{tab:etacha_syserr}.

\begin{table}[!htbp]
 \begin{center}
 \begin{small}
 \caption{\label{tab:etacha_syserr} The systematic errors of the matrix elements.}
 \begin{tabular}{ccccccccc}\hline\hline
  	Source 		        & a(\%)  & b(\%) & c(\%)   & d(\%)  & e(\%)   & f(\%)  & $\alpha(\eta\ra\pio\pio\pio)(\%)$ & $\alpha(\etap\ra\pio\pio\pio)(\%)$  \\ \hline
   	Fitting bias        & 0.63   & 2.08  & 44.93   & 1.60   & 24.01   & 7.69   & 6.00   & 2.88          \\
   	Efficiency Para.    & 1.30   & 2.87  & 485.85  & 7.91   & 13.07  & 14.65  & 0.41   & 0.90   \\
   	Tracking efficiency & 0.08   & 0.61  & 101.01  & 0.16   & 2.91    & 0.10   &  -     &  -   \\
    $\pio$ efficiency 	& 0.05   & 2.01  & 11.85   & 1.66   & 2.99    & 1.3    & 3.67   & 1.31     \\
  	$\eta$ constraint   & 0.38   & 1.60  & 794.19  & 2.07   & 12.26   & 12.93  & 3.84   &  -         \\
    Resolution          & 0.01   & 0.22  & 3.42    & 0.04   & 0.05    & 0.05   &  -     &  -          \\
   	Fit Range           &  -     &  -    & -       & -      & -       & -      & 3.74   & 3.64   \\
    Peaking bkg.\ shape  &  -     &  -    & -       & -      & -       & -      &  -     & 0.35   \\
    Peaking bkg.\ Br.    &  -     &  -    & -       & -      & -       & -      &  -     & 0.52   \\
    Non-Peaking bkg.    &  -     &  -    & -       & -      & -       & -      &  -     & 1.45   \\
	\hline
   Total    & 1.49   & 4.43  & 937.64  & 8.49   & 30.25   & 21.04  & 8.86   & 5.16   \\
\hline\hline
\end{tabular}
\end{small}
\end{center}
\end{table}

\section{SUMMARY}
Using the large $\jpsi$ sample (1.3 billion $\jpsi$ events) collected with BESIII, the matrix elements for Dalitz plots of decays $\eta\ra\pip\pim\pio$, $\eta\ra\pio\pio\pio$, $\etap\ra\pio\pio\pio$ have been measured.

For decay $\eta\ra\pip\pim\pio$, our fit gives the results
\begin{small}
\begin{eqnarray}
	a & = & -1.128 \pm 0.016 \pm 0.017\nonumber\\
	b & = &  0.154 \pm 0.017 \pm 0.007\nonumber\\
	c & = &  (0.058 \pm 0.85 \pm 0.54)\times10^{-2}\nonumber\\
	d & = &  0.085 \pm 0.016 \pm 0.007\nonumber\\
	e & = &  0.017 \pm 0.019 \pm 0.005\nonumber\\
	f & = &  0.173 \pm 0.031 \pm 0.036
\end{eqnarray}
\end{small}
with $\chisq/NDF=1400/1153$, where the first errors are statical and the second systematic. Table~\ref{tab:etacha_otherexp} shows the experimental and theoretical values of the matrix elements for $\eta\ra\pip\pim\pio$. In our fitting results, the c and e are consistent with zero as charge conjugation expected, and the other parameters are in reasonable agreement with the previous measurements of other groups, while there is a small but significant (about 2 s.d.) disagreement with respect to the calculation at next-to-next-to-leading order in Chiral Perturbation Theory in Ref.~\cite{ref:nnlo}, as Table~\ref{tab:etacha_otherexp} listed. (The statistical and systematic errors are added in quadrature.)

\begin{table}[!htbp]
 \begin{center}
 \begin{small}
 \caption{\label{tab:etacha_otherexp} Theoretical and experimental values of the matrix elements for $\eta\ra\pip\pim\pio$.}
 \begin{tabular}{ccccc}\hline\hline
  	Theory/Exp. & a 	& b  & 	d	 &	 f	 \\\hline
   	ChPT NNLO\cite{ref:chpt2loop}  & $-1.271 \pm 0.075$ & $0.394 \pm 0.102$ & $0.055 \pm 0.057$ & $0.025 \pm 0.160$ \\
   	CBAR\cite{ref:cbar}  &$ -1.22\pm0.07 $&$ 0.22\pm0.11 $&$ 0.06(fixed) $&$ - $\\
   	KLOE\cite{ref:kloe}  &$ -1.09^{+0.013}_{- 0.024} $&$ 0.124\pm0.016 $&$ 0.057^{+0.016}_{-0.022} $&$ 0.14\pm0.03$\\
   	This work  &$ -1.128\pm0.024 $&$ 0.154\pm0.019 $&$ 0.085\pm0.020 $&$ 0.173\pm0.048$\\
\hline\hline
\end{tabular}
\end{small}
\end{center}
\end{table}

For the neutral decays, the result is $\alpha(\eta\ra\pio\pio\pio)$ = -0.055 $\pm$ 0.014 $\pm$ 0.005 and $\alpha(\etap\ra\pio\pio\pio)$ = -0.687 $\pm$ 0.049 $\pm$ 0.036, respectively.
Comparison of our results with some other groups' and theoretical calculation is showed in Table~\ref{tab:etaneu_expresults}
and Table~\ref{tab:etapneu_expresults}. For decay $\eta\ra\pio\pio\pio$, our result is in reasonable agreement with world other experiments
and for the result of decay $\etap\ra\pio\pio\pio$, our result is consistent with the result of GAMS\@.
There is also a small disagreement with respect to the ChPT predictions in the decay $\eta \ra \pio\pio\pio$.\\

\begin{table}[!htbp]
\begin{center}
\begin{small}
\caption{\label{tab:etaneu_expresults}Theoretical and experimental values of the matrix element for $\eta\ra\pio\pio\pio$}
 \begin{tabular}{cc}\hline\hline
  	Theory/Exp. & $\alpha$ \\\hline
  	ChPT/NLO\cite{ref:etaneuoneloop} & 0.015\\
  	ChPT/NLO + unit.correction\cite{ref:etaneuwithcor} & (-0.014)-(-0.007)\\	
  	UChPT Fit + Bethe Salpeter\cite{ref:alphachpt3} & $-0.031 \pm 0.003$\\
  	ChPT/NNLO\cite{ref:nnlo} & $0.013 \pm 0.032$\\
   	KLOE\cite{ref:etaneukloe}  & $-0.0301 \pm 0.0035^{+0.0022}_{-0.0035}$\\
   	WASA\cite{ref:etaneuwasa}  & $-0.027 \pm 0.008 \pm 0.005$\\
   	CRYB\cite{ref:etaneucryb}  & $-0.0322 \pm 0.0012 \pm 0.0022$\\
   	SND\cite{ref:etaneusnd}  & $-0.010 \pm 0.021 \pm 0.010 $\\
   	CBAR\cite{ref:etaneucbar}  & $-0.052 \pm 0.017 \pm 0.010$\\
   	GAM2\cite{ref:etaneugam2}  & $-0.022 \pm 0.023$\\
   	This work  & $-0.055 \pm 0.014 \pm 0.005$\\
\hline\hline
\end{tabular}
\end{small}
\end{center}
\end{table}

\begin{table}[!htbp]
\begin{center}
\begin{small}
\caption{\label{tab:etapneu_expresults}Theoretical and experimental values of the matrix element for $\eta'\ra\pio\pio\pio$}
\begin{tabular}{cc}\hline\hline
  	Theory/Exp. & $\alpha$ \\\hline
  	ChPT\cite{ref:etapneuchpt} & ($0.1\pm1.7$) to ($-2.7\pm1.0$)\\
   	GAMS\cite{ref:etapneugam} & $-0.59 \pm 0.18$\\
   	GAM2\cite{ref:etapneugam2} & $-0.1 \pm 0.3$ \\
   	This work  & $-0.687 \pm 0.049 \pm 0.036$\\
\hline\hline
\end{tabular}
\end{small}
\end{center}
\end{table}

\begin{thebibliography}{9}
\bibitem{ref:chpt} D.G. Sutherland, Phys. Lett. {\bf 23}, 384 (1966).%doi:10.1016/0031-9163(66)90477-X
\bibitem{ref:kloe} F. Ambrosino {\it et al.} (KLOE Collaboration), JHEP {\bf 0805}, 006 (2008).
\bibitem{ref:etaneukloe} F. Ambrosino {\it et al.} (KLOE Collaboration), Phys. Lett. {\bf B694}, 16 (2010).
\bibitem{ref:cbar} A. Abele {\it et al.} (Crystal Barrel Collaboration), Phys. Lett. B {\bf 417}, 197 (1998).
\bibitem{ref:etaneucbar} A. Abele {\it et al.} (Crystal Barrel Collaboration), Phys. Lett. {\bf B417}, 193 (1998).
%doi:10.1016/S0370-2693(97)01377-4
%\bibitem{ref:etaneucryb} S. Prakhov {\it et al.} (Crystal Ball Collaboration), Phys. Rev. {\bf C79}, 035204 (2009).%doi:10.1103/PhysRevC.79.035204
\bibitem{ref:aspk} J. G. Layter, J. A. Appel, A. Kotlewski, W. Lee, S. Stein, and J. J. Thaler, Phys. Rev. Lett. {\bf 29}, 316 (1972) %doi:10.1103/PhysRevLett.29.316
\bibitem{ref:etapneugam2} D. Alde {\it et al.} (LAPP Collaboration),  Z. Phys. {\bf C36} 603 (1987).%doi:10.1007/BF01630597
\bibitem{ref:etapneugam} A. M. Blik {\it et al.} Phys. Atom. Nucl. {\bf 71}, 2124 (2008).%doi:10.1134/S1063778808120144
%\bibitem{Alde:1987jt} A.M. Blik {\it et al.}, Phys. Atom. Nucl. {\bf 71}(12), 2124 (2008).%doi:10.1134/S1063778808120144
%\bibitem{Blik:2008zz} D. Alde {\it et al.}, Phys. Atom. Nucl. {\bf 36}(4), 603 (1987).%doi:10.1007/BF01630597
\bibitem{bepc2_design} J. Z. Bai {\it et al.} (BES Collaboration), Nucl. Instrum. Methods Phys. Res. A {\bf 458}, 627 (2001).
\bibitem{Ablikim2010345} M.~Ablikim {\it et al.} (BESIII Collaboration), Nucl. Instrum. Methods Phys. Res., Sect. A {\bf 614}(3), 345 (2010).
%\bibitem{{3136260} S. Agostinelli {\it et al.} (\textsc{geant}{\footnotesize
%4} Collaboration), Nucl. Instrum. Methods Phys. Res. A {\bf 506}, 250 (2003).
\bibitem{ref:boost} Z. Y. Deng {\it et al.}, Chinese Phys. C {\bf 30}, 371 (2006).
\bibitem{ref:kkmc} S. Jadach and B.F.L. Ward and Z. Was, Comput. Phys. Commun. {\bf 130}, 260 (2009).
\bibitem{ref:kkmc2} S. Jadach and B.F.L. Ward and Z. Was, Phys. Rev. D {\bf 63}, 113009 (2001).
\bibitem{ref:evtgen} R. G. Ping {\it et al.}, Chinese Phys. C {\bf 32}, 599 (2008).
\bibitem{ref:pdg} K.A. Olive {\it et al.} (Particle Data Group), Chin. Phys. C, {\bf 38}, 090001 (2014).
\bibitem{ref:lundcharm} J. C. Chen, G. S. Huang, X. R. Qi, D. H. Zhang, and Y. S.
Zhu, Phys. Rev. D {\bf 62}, 034003 (2000).
\bibitem{ref:boss} W. D. Li et al., Proceeding of CHEP06, Mumbai, India, 2006.
\bibitem{ref:etapipi} A. M. Blik {\it et al.}, Phys. Atom. Nucl. {\bf 72}, 231 (2009).%doi:10.1134/S1063778809020045
\bibitem{ref:chpt2loop} J. Bijnens and K. Ghorbani, e-Print Archive:arXiv:0709.0230 [hep-ph].%doi:10.1088/1126-6708/2007/11/030
%\bibitem{ref:cbar} A. Abele {\it et al.} (Crystal Barrel Collaboration), Phys. Lett. B {\bf 417}, 197 (1998). %doi:10.1016/S0370-2693(97)01376-2
%\bibitem{ref:kloe} F. Ambrosino {\it et al.} (KLOE Collaboration), JHEP {\bf 0805}, 006 (2008). %doi:10.1088/1126-6708/2008/05/006
\bibitem{ref:etaneuoneloop} J. Gasser and H. Leutwyler, Nucl. Phys. B {\bf 250}, 539 (1985).
\bibitem{ref:etaneuwithcor} J. Kambor and C. Wiesendanger and D. Wyler, Nucl. Phys. B {\bf 465}, 215 (1996).
\bibitem{ref:alphachpt3} B. Borasoy, R. Ni?ler, EPJA {\bf 19}, 367 (2004).%10.1140/epja/i2003-10140-1
\bibitem{ref:nnlo} J. Bijnens, K. Ghorbani, JHEP {\bf 0711}, 030 (2007) [arXiv:0709.0230l [hep-ph].
%\bibitem{ref:etaneukloe} F. Ambrosino {\it et al.} (KLOE Collaboration), Phys. Lett. {\bf B694}, 16 (2010).
\bibitem{ref:etaneuwasa} C. Adolph {\it et al.} (WASA Collaboration), Phys. Lett {\bf B677}, 24 (2009).%doi:10.1016/j.physletb.2009.03.063
\bibitem{ref:etaneucryb} S. Prakhov {\it et al.} (Crystal Ball Collaboration), Phys. Rev. {\bf C79}, 035204 (2009).%doi:10.1103/PhysRevC.79.035204
\bibitem{ref:etaneusnd} M. N. Achasov {\it et al.}, JETP Lett. {\bf 73}, 451 (2001).%doi:10.1134/1.1385655
%\bibitem{ref:etaneucbar} A. Abele {\it et al.} (Crystal Barrel Collaboration), Phys. Lett. {\bf B417}, 193 (1998).
%doi:10.1016/S0370-2693(97)01377-4
\bibitem{ref:etaneugam2} D. Alde {\it et al.} (LAPP Collaboration), Z. Phys. {\bf C25}, 225 (1984).
\bibitem{ref:etapneuchpt} B. Borasoy, R. Nissler, Eur. Phys. J. {\bf A26}, 383 (2005).%doi:10.1140/epja/i2005-10188-9
\end{thebibliography}

%\bibliographystyle{unsrt}
%\bibliography{eta3pi.bib}
\end{document}
