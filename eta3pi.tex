%%%%%%%%%%%%%%%%%%%%%%%%%%%%%%%%%%%%%%%%%%%%%
%%%%%                                   %%%%%
%%%%%       2014.8.29,   afternoon      %%%%%
%%%%%          1.1st version            %%%%%
%%%%%                                   %%%%%
%%%%%%%%%%%%%%%%%%%%%%%%%%%%%%%%%%%%%%%%%%%%%
\RequirePackage{lineno}
\documentclass[prd,amsmath,amssymb,showpacs,superscriptaddress,nofootinbib,twocolumn]{revtex4}
\usepackage{graphicx}
\usepackage{epsfig}
\usepackage{dcolumn}
\usepackage{bm}
\usepackage{overpic}
\setlength{\parskip}{0\baselineskip}
\newcommand{\ra}{\rightarrow}
\newcommand{\BR}{{\cal B}}
\newcommand{\eff}{\varepsilon}
\newcommand{\LL}{\ell^+\ell^-}
\newcommand{\jpsi}{J/\psi}
\newcommand{\pio}{\pi^{0}}
\newcommand{\pip}{\pi^{+}}
\newcommand{\pim}{\pi^{-}}
\newcommand{\etap}{\eta^{\prime}}
\newcommand{\chisq}{\chi^{2}}
\renewcommand{\thefootnote}{\fnsymbol{footnote}}
\pagewiselinenumbers
\begin{document}

\normalsize
\parskip=0pt plus 1pt minus 1pt

\title{\boldmath Dalitz plot analyses of $\eta/\etap \ra 3\pi$  }

\author{
M.~Ablikim$^{1}$, M.~N.~Achasov$^{8,a}$, X.~C.~Ai$^{1}$, O.~Albayrak$^{4}$, M.~Albrecht$^{3}$, D.~J.~Ambrose$^{42}$, A.~Amoroso$^{46A,46C}$, F.~F.~An$^{1}$, Q.~An$^{43}$, J.~Z.~Bai$^{1}$, R.~Baldini Ferroli$^{19A}$, Y.~Ban$^{29}$, D.~W.~Bennett$^{18}$, J.~V.~Bennett$^{4}$, M.~Bertani$^{19A}$, D.~Bettoni$^{20A}$, J.~M.~Bian$^{41}$, F.~Bianchi$^{46A,46C}$, E.~Boger$^{22,g}$, O.~Bondarenko$^{23}$, I.~Boyko$^{22}$, S.~Braun$^{38}$, R.~A.~Briere$^{4}$, H.~Cai$^{48}$, X.~Cai$^{1}$, O. ~Cakir$^{37A}$, A.~Calcaterra$^{19A}$, G.~F.~Cao$^{1}$, S.~A.~Cetin$^{37B}$, J.~F.~Chang$^{1}$, G.~Chelkov$^{22,b}$, G.~Chen$^{1}$, H.~S.~Chen$^{1}$, J.~C.~Chen$^{1}$, M.~L.~Chen$^{1}$, S.~J.~Chen$^{27}$, X.~Chen$^{1}$, X.~R.~Chen$^{24}$, Y.~B.~Chen$^{1}$, H.~P.~Cheng$^{16}$, X.~K.~Chu$^{29}$, Y.~P.~Chu$^{1}$, G.~Cibinetto$^{20A}$, D.~Cronin-Hennessy$^{41}$, H.~L.~Dai$^{1}$, J.~P.~Dai$^{1}$, D.~Dedovich$^{22}$, Z.~Y.~Deng$^{1}$, A.~Denig$^{21}$, I.~Denysenko$^{22}$, M.~Destefanis$^{46A,46C}$, F.~De~Mori$^{46A,46C}$, Y.~Ding$^{25}$, C.~Dong$^{28}$, J.~Dong$^{1}$, L.~Y.~Dong$^{1}$, M.~Y.~Dong$^{1}$, S.~X.~Du$^{50}$, J.~Z.~Fan$^{36}$, J.~Fang$^{1}$, S.~S.~Fang$^{1}$, Y.~Fang$^{1}$, L.~Fava$^{46B,46C}$, F.~Feldbauer$^{21}$, G.~Felici$^{19A}$, C.~Q.~Feng$^{43}$, E.~Fioravanti$^{20A}$, C.~D.~Fu$^{1}$, Q.~Gao$^{1}$, Y.~Gao$^{36}$, I.~Garzia$^{20A}$, C.~Geng$^{43}$, K.~Goetzen$^{9}$, W.~X.~Gong$^{1}$, W.~Gradl$^{21}$, M.~Greco$^{46A,46C}$, M.~H.~Gu$^{1}$, Y.~T.~Gu$^{11}$, Y.~H.~Guan$^{1}$, L.~B.~Guo$^{26}$, T.~Guo$^{26}$, Y.~P.~Guo$^{21}$, Z.~Haddadi$^{23}$, S.~Han$^{48}$, Y.~L.~Han$^{1}$, F.~A.~Harris$^{40}$, K.~L.~He$^{1}$, M.~He$^{1}$, Z.~Y.~He$^{28}$, T.~Held$^{3}$, Y.~K.~Heng$^{1}$, Z.~L.~Hou$^{1}$, C.~Hu$^{26}$, H.~M.~Hu$^{1}$, J.~F.~Hu$^{46A}$, T.~Hu$^{1}$, G.~M.~Huang$^{5}$, G.~S.~Huang$^{43}$, H.~P.~Huang$^{48}$, J.~S.~Huang$^{14}$, X.~T.~Huang$^{31}$, Y.~Huang$^{27}$, T.~Hussain$^{45}$, Q.~Ji$^{1}$, Q.~P.~Ji$^{28}$, X.~B.~Ji$^{1}$, X.~L.~Ji$^{1}$, L.~L.~Jiang$^{1}$, L.~W.~Jiang$^{48}$, X.~S.~Jiang$^{1}$, J.~B.~Jiao$^{31}$, Z.~Jiao$^{16}$, D.~P.~Jin$^{1}$, S.~Jin$^{1}$, T.~Johansson$^{47}$, A.~Julin$^{41}$, N.~Kalantar-Nayestanaki$^{23}$, X.~L.~Kang$^{1}$, X.~S.~Kang$^{28}$, M.~Kavatsyuk$^{23}$, B.~C.~Ke$^{4}$, B.~Kloss$^{21}$, O.~B.~Kolcu$^{37B,c}$, B.~Kopf$^{3}$, M.~Kornicer$^{40}$, W.~Kuehn$^{38}$, A.~Kupsc$^{47}$, W.~Lai$^{1}$, J.~S.~Lange$^{38}$, M.~Lara$^{18}$, P. ~Larin$^{13}$, M.~Leyhe$^{3}$, Cheng~Li$^{43}$, Cui~Li$^{43}$, D.~M.~Li$^{50}$, F.~Li$^{1}$, G.~Li$^{1}$, H.~B.~Li$^{1}$, J.~C.~Li$^{1}$, Jin~Li$^{30}$, K.~Li$^{12}$, K.~Li$^{31}$, Q.~J.~Li$^{1}$, T. ~Li$^{31}$, W.~D.~Li$^{1}$, W.~G.~Li$^{1}$, X.~L.~Li$^{31}$, X.~N.~Li$^{1}$, X.~Q.~Li$^{28}$, Z.~B.~Li$^{35}$, H.~Liang$^{43}$, Y.~F.~Liang$^{33}$, Y.~T.~Liang$^{38}$, D.~X.~Lin$^{13}$, B.~J.~Liu$^{1}$, C.~L.~Liu$^{4}$, C.~X.~Liu$^{1}$, F.~H.~Liu$^{32}$, Fang~Liu$^{1}$, Feng~Liu$^{5}$, H.~B.~Liu$^{11}$, H.~H.~Liu$^{15}$, H.~M.~Liu$^{1}$, J.~Liu$^{1}$, J.~P.~Liu$^{48}$, K.~Liu$^{36}$, K.~Y.~Liu$^{25}$, Q.~Liu$^{39}$, S.~B.~Liu$^{43}$, X.~Liu$^{24}$, X.~X.~Liu$^{39}$, Y.~B.~Liu$^{28}$, Z.~A.~Liu$^{1}$, Zhiqiang~Liu$^{1}$, Zhiqing~Liu$^{21}$, H.~Loehner$^{23}$, X.~C.~Lou$^{1,d}$, H.~J.~Lu$^{16}$, J.~G.~Lu$^{1}$, R.~Q.~Lu$^{17}$, Y.~Lu$^{1}$, Y.~P.~Lu$^{1}$, C.~L.~Luo$^{26}$, M.~X.~Luo$^{49}$, T.~Luo$^{40}$, X.~L.~Luo$^{1}$, M.~Lv$^{1}$, X.~R.~Lyu$^{39}$, F.~C.~Ma$^{25}$, H.~L.~Ma$^{1}$, Q.~M.~Ma$^{1}$, S.~Ma$^{1}$, X.~Y.~Ma$^{1}$, F.~E.~Maas$^{13}$, M.~Maggiora$^{46A,46C}$, Q.~A.~Malik$^{45}$, Y.~J.~Mao$^{29}$, Z.~P.~Mao$^{1}$, S.~Marcello$^{46A,46C}$, J.~G.~Messchendorp$^{23}$, J.~Min$^{1}$, T.~J.~Min$^{1}$, R.~E.~Mitchell$^{18}$, X.~H.~Mo$^{1}$, Y.~J.~Mo$^{5}$, H.~Moeini$^{23}$, C.~Morales Morales$^{13}$, K.~Moriya$^{18}$, N.~Yu.~Muchnoi$^{8,a}$, H.~Muramatsu$^{41}$, Y.~Nefedov$^{22}$, F.~Nerling$^{13}$, I.~B.~Nikolaev$^{8,a}$, Z.~Ning$^{1}$, S.~Nisar$^{7}$, X.~Y.~Niu$^{1}$, S.~L.~Olsen$^{30}$, Q.~Ouyang$^{1}$, S.~Pacetti$^{19B}$, P.~Patteri$^{19A}$, M.~Pelizaeus$^{3}$, H.~P.~Peng$^{43}$, K.~Peters$^{9}$, J.~L.~Ping$^{26}$, R.~G.~Ping$^{1}$, R.~Poling$^{41}$, Y.~N.~Pu$^{17}$, M.~Qi$^{27}$, S.~Qian$^{1}$, C.~F.~Qiao$^{39}$, L.~Q.~Qin$^{31}$, N.~Qin$^{48}$, Y.~Qin$^{29}$, Z.~H.~Qin$^{1}$, J.~F.~Qiu$^{1}$, K.~H.~Rashid$^{45}$, C.~F.~Redmer$^{21}$, H.~L.~Ren$^{17}$, M.~Ripka$^{21}$, G.~Rong$^{1}$, X.~D.~Ruan$^{11}$, V.~Santoro$^{20A}$, A.~Sarantsev$^{22,e}$, M.~Savri��$^{20B}$, K.~Schoenning$^{47}$, S.~Schumann$^{21}$, W.~Shan$^{29}$, M.~Shao$^{43}$, C.~P.~Shen$^{2}$, X.~Y.~Shen$^{1}$, H.~Y.~Sheng$^{1}$, M.~R.~Shepherd$^{18}$, W.~M.~Song$^{1}$, S.~Spataro$^{46A,46C}$, B.~Spruck$^{38}$, S.~Stefano$^{46A,46C}$, G.~X.~Sun$^{1}$, J.~F.~Sun$^{14}$, S.~S.~Sun$^{1}$, Y.~J.~Sun$^{43}$, Y.~Z.~Sun$^{1}$, Z.~J.~Sun$^{1}$, Z.~T.~Sun$^{43}$, C.~J.~Tang$^{33}$, X.~Tang$^{1}$, I.~Tapan$^{37C}$, E.~H.~Thorndike$^{42}$, M.~Tiemens$^{23}$, D.~Toth$^{41}$, M.~Ullrich$^{38}$, I.~Uman$^{37B}$, G.~S.~Varner$^{40}$, B.~Wang$^{28}$, B.~L.~Wang$^{39}$, D.~Wang$^{29}$, D.~Y.~Wang$^{29}$, K.~Wang$^{1}$, L.~L.~Wang$^{1}$, L.~S.~Wang$^{1}$, M.~Wang$^{31}$, P.~Wang$^{1}$, P.~L.~Wang$^{1}$, Q.~J.~Wang$^{1}$, S.~G.~Wang$^{29}$, W.~Wang$^{1}$, X.~F. ~Wang$^{36}$, Y.~D.~Wang$^{19A}$, Y.~F.~Wang$^{1}$, Y.~Q.~Wang$^{21}$, Z.~Wang$^{1}$, Z.~G.~Wang$^{1}$, Z.~H.~Wang$^{43}$, Z.~Y.~Wang$^{1}$, D.~H.~Wei$^{10}$, J.~B.~Wei$^{29}$, P.~Weidenkaff$^{21}$, S.~P.~Wen$^{1}$, M.~Werner$^{38}$, U.~Wiedner$^{3}$, M.~Wolke$^{47}$, L.~H.~Wu$^{1}$, N.~Wu$^{1}$, Z.~Wu$^{1}$, L.~G.~Xia$^{36}$, Y.~Xia$^{17}$, D.~Xiao$^{1}$, Z.~J.~Xiao$^{26}$, Y.~G.~Xie$^{1}$, Q.~L.~Xiu$^{1}$, G.~F.~Xu$^{1}$, L.~Xu$^{1}$, Q.~J.~Xu$^{12}$, Q.~N.~Xu$^{39}$, X.~P.~Xu$^{34}$, Z.~Xue$^{1}$, L.~Yan$^{43}$, W.~B.~Yan$^{43}$, W.~C.~Yan$^{43}$, Y.~H.~Yan$^{17}$, H.~X.~Yang$^{1}$, L.~Yang$^{48}$, Y.~Yang$^{5}$, Y.~X.~Yang$^{10}$, H.~Ye$^{1}$, M.~Ye$^{1}$, M.~H.~Ye$^{6}$, B.~X.~Yu$^{1}$, C.~X.~Yu$^{28}$, H.~W.~Yu$^{29}$, J.~S.~Yu$^{24}$, C.~Z.~Yuan$^{1}$, W.~L.~Yuan$^{27}$, Y.~Yuan$^{1}$, A.~Yuncu$^{37B,f}$, A.~A.~Zafar$^{45}$, A.~Zallo$^{19A}$, S.~L.~Zang$^{27}$, Y.~Zeng$^{17}$, B.~X.~Zhang$^{1}$, B.~Y.~Zhang$^{1}$, C.~Zhang$^{27}$, C.~C.~Zhang$^{1}$, D.~H.~Zhang$^{1}$, H.~H.~Zhang$^{35}$, H.~Y.~Zhang$^{1}$, J.~J.~Zhang$^{1}$, J.~Q.~Zhang$^{1}$, J.~W.~Zhang$^{1}$, J.~Y.~Zhang$^{1}$, J.~Z.~Zhang$^{1}$, S.~H.~Zhang$^{1}$, X.~J.~Zhang$^{1}$, X.~Y.~Zhang$^{31}$, Y.~Zhang$^{1}$, Y.~H.~Zhang$^{1}$, Z.~H.~Zhang$^{5}$, Z.~P.~Zhang$^{43}$, Z.~Y.~Zhang$^{48}$, J.~W.~Zhao$^{1}$, Lei~Zhao$^{43}$, Ling~Zhao$^{1}$, M.~G.~Zhao$^{28}$, Q.~Zhao$^{1}$, Q.~W.~Zhao$^{1}$, S.~J.~Zhao$^{50}$, T.~C.~Zhao$^{1}$, Y.~B.~Zhao$^{1}$, Z.~G.~Zhao$^{43}$, A.~Zhemchugov$^{22,g}$, B.~Zheng$^{44}$, J.~P.~Zheng$^{1}$, Y.~H.~Zheng$^{39}$, B.~Zhong$^{26}$, L.~Zhou$^{1}$, Li~Zhou$^{28}$, X.~Zhou$^{48}$, X.~R.~Zhou$^{43}$, X.~Y.~Zhou$^{1}$, K.~Zhu$^{1}$, K.~J.~Zhu$^{1}$, X.~L.~Zhu$^{36}$, Y.~C.~Zhu$^{43}$, Y.~S.~Zhu$^{1}$, Z.~A.~Zhu$^{1}$, J.~Zhuang$^{1}$, B.~S.~Zou$^{1}$, J.~H.~Zou$^{1}$
\\
\vspace{0.2cm}
(BESIII Collaboration)\\
\vspace{0.2cm} {\it
$^{1}$ Institute of High Energy Physics, Beijing 100049, People's Republic of China\\
$^{2}$ Beihang University, Beijing 100191, People's Republic of China\\
$^{3}$ Bochum Ruhr-University, D-44780 Bochum, Germany\\
$^{4}$ Carnegie Mellon University, Pittsburgh, Pennsylvania 15213, USA\\
$^{5}$ Central China Normal University, Wuhan 430079, People's Republic of China\\
$^{6}$ China Center of Advanced Science and Technology, Beijing 100190, People's Republic of China\\
$^{7}$ COMSATS Institute of Information Technology, Lahore, Defence Road, Off Raiwind Road, 54000 Lahore, Pakistan\\
$^{8}$ G.I. Budker Institute of Nuclear Physics SB RAS (BINP), Novosibirsk 630090, Russia\\
$^{9}$ GSI Helmholtzcentre for Heavy Ion Research GmbH, D-64291 Darmstadt, Germany\\
$^{10}$ Guangxi Normal University, Guilin 541004, People's Republic of China\\
$^{11}$ GuangXi University, Nanning 530004, People's Republic of China\\
$^{12}$ Hangzhou Normal University, Hangzhou 310036, People's Republic of China\\
$^{13}$ Helmholtz Institute Mainz, Johann-Joachim-Becher-Weg 45, D-55099 Mainz, Germany\\
$^{14}$ Henan Normal University, Xinxiang 453007, People's Republic of China\\
$^{15}$ Henan University of Science and Technology, Luoyang 471003, People's Republic of China\\
$^{16}$ Huangshan College, Huangshan 245000, People's Republic of China\\
$^{17}$ Hunan University, Changsha 410082, People's Republic of China\\
$^{18}$ Indiana University, Bloomington, Indiana 47405, USA\\
$^{19}$ (A)INFN Laboratori Nazionali di Frascati, I-00044, Frascati, Italy; (B)INFN and University of Perugia, I-06100, Perugia, Italy\\
$^{20}$ (A)INFN Sezione di Ferrara, I-44122, Ferrara, Italy; (B)University of Ferrara, I-44122, Ferrara, Italy\\
$^{21}$ Johannes Gutenberg University of Mainz, Johann-Joachim-Becher-Weg 45, D-55099 Mainz, Germany\\
$^{22}$ Joint Institute for Nuclear Research, 141980 Dubna, Moscow region, Russia\\
$^{23}$ KVI-CART, University of Groningen, NL-9747 AA Groningen, The Netherlands\\
$^{24}$ Lanzhou University, Lanzhou 730000, People's Republic of China\\
$^{25}$ Liaoning University, Shenyang 110036, People's Republic of China\\
$^{26}$ Nanjing Normal University, Nanjing 210023, People's Republic of China\\
$^{27}$ Nanjing University, Nanjing 210093, People's Republic of China\\
$^{28}$ Nankai University, Tianjin 300071, People's Republic of China\\
$^{29}$ Peking University, Beijing 100871, People's Republic of China\\
$^{30}$ Seoul National University, Seoul, 151-747 Korea\\
$^{31}$ Shandong University, Jinan 250100, People's Republic of China\\
$^{32}$ Shanxi University, Taiyuan 030006, People's Republic of China\\
$^{33}$ Sichuan University, Chengdu 610064, People's Republic of China\\
$^{34}$ Soochow University, Suzhou 215006, People's Republic of China\\
$^{35}$ Sun Yat-Sen University, Guangzhou 510275, People's Republic of China\\
$^{36}$ Tsinghua University, Beijing 100084, People's Republic of China\\
$^{37}$ (A)Ankara University, Dogol Caddesi, 06100 Tandogan, Ankara, Turkey; (B)Dogus University, 34722 Istanbul, Turkey; (C)Uludag University, 16059 Bursa, Turkey\\
$^{38}$ Universitaet Giessen, D-35392 Giessen, Germany\\
$^{39}$ University of Chinese Academy of Sciences, Beijing 100049, People's Republic of China\\
$^{40}$ University of Hawaii, Honolulu, Hawaii 96822, USA\\
$^{41}$ University of Minnesota, Minneapolis, Minnesota 55455, USA\\
$^{42}$ University of Rochester, Rochester, New York 14627, USA\\
$^{43}$ University of Science and Technology of China, Hefei 230026, People's Republic of China\\
$^{44}$ University of South China, Hengyang 421001, People's Republic of China\\
$^{45}$ University of the Punjab, Lahore-54590, Pakistan\\
$^{46}$ (A)University of Turin, I-10125, Turin, Italy; (B)University of Eastern Piedmont, I-15121, Alessandria, Italy; (C)INFN, I-10125, Turin, Italy\\
$^{47}$ Uppsala University, Box 516, SE-75120 Uppsala, Sweden\\
$^{48}$ Wuhan University, Wuhan 430072, People's Republic of China\\
$^{49}$ Zhejiang University, Hangzhou 310027, People's Republic of China\\
$^{50}$ Zhengzhou University, Zhengzhou 450001, People's Republic of China\\
\vspace{0.2cm}
$^{a}$ Also at the Novosibirsk State University, Novosibirsk, 630090, Russia\\
$^{b}$ Also at the Moscow Institute of Physics and Technology, Moscow 141700, Russia and at the Functional Electronics Laboratory, Tomsk State University, Tomsk, 634050, Russia \\
$^{c}$ Currently at Istanbul Arel University, Kucukcekmece, Istanbul, Turkey\\
$^{d}$ Also at University of Texas at Dallas, Richardson, Texas 75083, USA\\
$^{e}$ Also at the PNPI, Gatchina 188300, Russia\\
$^{f}$ Also at Bogazici University, 34342 Istanbul, Turkey\\
$^{g}$ Also at the Moscow Institute of Physics and Technology, Moscow 141700, Russia\\
}
}

\date{\today}

\begin{abstract}
Dalitz plot analysis plays an important role in understanding dynamics of three-body decays. Using 1.3 billion $\jpsi$ events accumulated in 2009 and 2012 with the BESIII detector, the Dalitz plot analyses
of decays $\eta \ra \pip\pim\pio$ and $\eta/\etap \ra \pio\pio\pio$ have been performed with $\jpsi \ra \gamma\eta/\etap$. For $\eta\to\pi^{+(0)}\pi^{-(0)}\pio$,
the measured Dalitz plot parameters are in reasonable agreement with the previous works.  The Dalitz
plot slope parameter for $\etap\ra\pio\pio\pio$ is also determined with significant
improvements on both statistic and systematic uncertainties.
\end{abstract}

%\pacs{13.25.Gv, 13.66.Bc, 14.40.Pq, 14.40.Rt}
\maketitle

\section{Introduction}
Measurements of matrix elements for particle decays help for obtaining deeper insight into dynamics of the processes
and into the structure of particles. Analyses of hadronic decays of $\eta/\eta'$ play important role in determining the mass
difference of u,d quark, testing Chiral Pertubation Theory(ChPT) and providing validation of models for
$\pi{}-\pi$ final state interaction. There have been a lot of groups had measured matrix elements of the decays $\eta\ra{}3\pi$
and our measurements give a reasonable agreement with those results. For decay of $\etap\ra{}3\pio$, two groups, the GAMS, GAM2 collaboration,
had measured related Dalitz plot parameter and the results showed large discrepancy.

In this article, with a new level of precision, we present results for the Dalitz plot parameter for $\etap\to\pio\pio\pio$ based on
about 1.3 billion $\jpsi$ events accumulated by BESIII at BEPCII.

\section{Detector and Monte Carlo simulation}
BEPCII is a double-ring $e^{+}e^{-}$ collider running at c.m. energy $\sqrt{s}$ = 2.0 - 4.6 GeV
and designed to provide a peak luminosity of $10^{33}$ cm$^{-2}s^{-1}$ at the c.m. energy of $3.770$ GeV.
The BESIII detector has a geometrical acceptance of $93\%$ of $4\pi$ and has four main components:
(1) A small-cell, helium-based ($40\%$ He, $60\%$ C$_{3}$H$_{8}$) main drift chamber (MDC) with $43$ layers
providing an average single-hit resolution of $135$ $\mu$m, and charged-particle momentum resolution in a $1$ T
magnetic field of $0.5\%$ at $1.0$ GeV$/c$. 
(2) A time-of-flight system (TOF) constructed of $5$ cm thick plastic scintillators,
with $176$ detectors of $2.4$ m length in two layers in the barrel and $96$ fan-shaped detectors in the endcaps.
The barrel (endcap) time resolution of $80$ ps ($110$ ps) provides $2\sigma$ $K/\pi$ separation for momenta up to $\sim1.0$ GeV$/c$.
(3) An electromagnetic calorimeter (EMC) consisting of $6240$ CsI(Tl) crystals in a cylindrical structure (barrel) and two endcaps.
The energy resolution at $1.0$ GeV$/c$ is $2.5\%$ ($5\%$) in the barrel (endcaps),
and the position resolution is $6$ mm ($9$ mm) in the barrel (endcaps).
(4) The muon system (MUC) consists of $1000$ m$^{2}$ of Resistive Plate Chambers (RPCs)
in nine barrel and eight endcap layers and provides $2.0$ cm position resolution.

The optimization of the selection criteria, determination of the detection efficiency and estimation
of potential backgrounds are performed through full simulated Monte Carlo (MC) samples.
The GEANT4-based simulation software BOOST includes geometric and material description of the
BESIII detector, detector response and digitization models as well as tracking of the detector
running condition and performance, is used to generate MC samples.
The production of $\jpsi$ resonance is simulated by the
Monte Carlo event generator KKMC, while the decays are generated by EvtGen
for known decay modes with branching ratios being set to the PDG world average
values, and by Lundcharm for the remaining unknown decays.

\section{Event Selection}
Charged-particle tracks are reconstructed from hits in MDC within the polar angle range
$|cos\theta| < 0.93$. Tracks that extrapolate to be within 10 cm of the interaction point
in the beam direction and 1 cm in the plane perpendicular to the beam are selected. 
Photon candidates are reconstructed by isolated showers in EMC. 
The photon energy is required to be at least 25 MeV in barrel ($|$cos$\theta|$ $<$ 0.80),
and 50 MeV in end-caps (0.86 $<$ $|$cos$\theta|$ $<$ 0.92).
To eliminate showers produced by charged particles, the angle between the shower and
nearest charged track must be greater than 10 degrees.
The photon with the maximum energy is regarded as from $\jpsi$. 

For $\jpsi\ra\gamma\pip\pim\pio$, the candidate events are required to have at least two charged tracks with 
zero net charge and the number of photons must be larger than two. The primary vertex fit is performed to the
$\pip\pim$ tracks and requried to be successful. A five-kinematic (5C) fit is performed where the constraints
are 4-momentum of $\jpsi$ and invariant mass of $\pio$. Events with smallest $\chi^{2}_{5C}$ are selected.
To reject possible backgrounds with two or four photons in the final states, the 4C-fit probability for
assignment $\jpsi\ra\pip\pim\gamma\gamma\gamma$ must be larger than those of $\jpsi\ra\pip\pim\gamma\gamma$
and $\jpsi\ra\pip\pim\gamma\gamma\gamma\gamma$.

For neutral decays, the candidate events are required to have no charged tracks. The $\pio\ra\gamma\gamma$ 
candidates are formed from pairs of photon candidates that are kinematically fit to $\pio$ mass, and the 
$\chisq$ from the kinematic fit with 1 degree of freedom are required to be less than 25. 
True $\pio$ mesons decay isotropically in the $\pio$ rest frame and their decay distributions are flat,
contrary to $\pio$ candidates originating from wrong photon combinations.
To avoid miss-combination of photons, the decay angle, defined as the polar angle of a photon in the 
$\pio$ rest frame, requires to be satisfy $|cos\theta_{decay}| < 0.95$.
Events with at least seven photons, which form at least three distinct $\pio$ candidates, are selected.
A 7C kinematic fit is performed to the $\jpsi\ra\gamma\pio\pio\pio$ (constraints are the 4-momentum of $\jpsi$
and the three $\pio$ masses) and $\chisq < 70$ is required.
For $\etap$ decay, $|m_{\gamma\pio}-m_{\omega}| > 0.05$ GeV/$c^{2}$ is required to veto background from
$\jpsi\ra\omega\pio\pio$. 
$\chisq(\gamma\pio\pio\pio) < \chisq(\gamma\eta\pio\pio)$, $|m_{\gamma\gamma} - m_{\eta}| > 0.03$ GeV/$c^{2}$ 
are used to remove as many as possible of peaking background events from $\eta'\ra\eta\gamma\gamma$.
%In order to suppress electronic noise and energy deposition unrelated to physical event,
%the EMC time $t$ of the photon candidate must be in coincidence with collision events
%in the range 0 $\leq$ $t$ $\leq$ 700 ns.
%
%A kinematic fitting that utilizes momentum and energy conservation (4C) is applied under
%hypothesis of $e^{+}e^{-} \to \gamma \gamma \ell^+\ell^-$ to improve mass resolution and reduce potential backgrounds.
%The chi-square ($\chi^{2}_{4C}$) of kinematic fit is required to be less than $40$.
%If there are more than two photons in a event, the combination of $\gamma \gamma \ell^{+}\ell^{-}$
%with least $\chi^{2}_{4C}$ is chosen.
%To suppress background events from radiative Bhabha and radiative dimuon events associated
%with a fake photon, the two selected photons are further required to be with energy larger
%than $0.08$ GeV.
%
%After all above requirements are imposed, scatter plots of the invariant mass of $\gamma\gamma$
%versus that of $\ell^+\ell^-$ for data at $\sqrt{s}$ = 4.230 and 4.260 GeV are shown in Fig.~\ref{scatter}.
%Clear accumulations of events are observed around the intersection of the $\eta$ and $J/\psi$ regions,
%which indicate $e^+e^-\to\eta J/\psi$ signals.
%There are no significant signals observed around the intersection of $\pi^0$ and $J/\psi$ regions.
%The corresponding invariant mass distributions of $\ell^+\ell^-$ are shown in Fig.~\ref{Mll}.
%MC studies show that dominant backgrounds are from radiative Bhabha and dimuon events,
%and are expected to be flat distributions around $J/\psi$ signal region.
%The high level background is observed in the $e^{+}e^{-}$ mode due to the much larger Bhabha background.
%The candidate event of $e^+e^-\to\eta J/\psi$ is required to be within $J/\psi$ signal region,
%defined as $3.067<M(\ell^{+}\ell^{-})<3.127$ GeV/$c^{2}$.
%A $J/\psi$ sideband region, $2.932<M(\ell^{+}\ell^{-})<3.052$ GeV/$c^{2}$
%and $3.142<M(\ell^{+}\ell^{-})<$ $3.262$ GeV/$c^{2}$,
%which is four times of the size of the signal region, is used to estimate contribution from background events in M($\gamma\gamma$).
%%%%%%Scatter Plots%%%%%%%%
%\begin{figure*}[htbp]
%\begin{center}
%\begin{overpic}[width=7.0cm,height=5.0cm,angle=0]{graph/hmgg_Jpsi_mumu_4230_2D.eps}
%\put(25,60){\large\bf (a)}
%\end{overpic}
%\begin{overpic}[width=7.0cm,height=5.0cm,angle=0]{graph/hmgg_Jpsi_ee_4230_2D.eps}
%\put(25,60){\large\bf (b)}
%\end{overpic}
%\begin{overpic}[width=7.0cm,height=5.0cm,angle=0]{graph/hmgg_Jpsi_mumu_4260_2D.eps}
%\put(25,60){\large\bf (c)}
%\end{overpic}
%\begin{overpic}[width=7.0cm,height=5.0cm,angle=0]{graph/hmgg_Jpsi_ee_4260_2D.eps}
%\put(25,60){\large\bf (d)}
%\end{overpic}
%\end{center}
%\caption{Scatter plots of $M(\gamma\gamma)$ versus $M(\ell^{+}\ell^{-})$ at $\sqrt{s}$ = 4.230 (a)(b) and 4.260 GeV (c)(d).
%The left two plots is for $\mu^{+}\mu^{-}$ mode and right plots for $e^{+}e^{-}$ mode.}\label{scatter}
%\end{figure*}
%%%%%%%%%%%%%
%%%%%%%%%%%%%%%%%%%%%
%\begin{figure*}[htbp]
%\begin{center}
%\begin{overpic}[width=7.0cm,height=5.0cm,angle=0]{graph/4230_Jpsi_mumu.eps}
%\put(25,60){\large\bf (a)}
%\end{overpic}
%\begin{overpic}[width=7.0cm,height=5.0cm,angle=0]{graph/4230_Jpsi_ee.eps}
%\put(25,60){\large\bf (b)}
%\end{overpic}
%\begin{overpic}[width=7.0cm,height=5.0cm,angle=0]{graph/4260_Jpsi_mumu.eps}
%\put(25,60){\large\bf (c)}
%\end{overpic}
%\begin{overpic}[width=7.0cm,height=5.0cm,angle=0]{graph/4260_Jpsi_ee.eps}
%\put(25,60){\large\bf (d)}
%\end{overpic}
%\end{center}
%\caption{The invariant mass distributions of lepton pairs at $\sqrt{s}$ = 4.230 (a)(b) and 4.260 GeV (c)(d).
%The left two plots is for $M(\mu^{+}\mu^{-})$ and right plots for $M(e^{+}e^{-})$.}\label{Mll}
%\end{figure*}
%%%%%%%%%%%%%
%
%With further $J/\psi$ mass window required, the invariant mass distributions of two photons
%$M(\gamma \gamma)$ are depicted in Fig.~\ref{4230gg} for data at $\sqrt{s}$ = 4.230 and 4.260 GeV,
%respectively. Clear $\eta$ signals are observed.
%The corresponding normalized distributions for the events in the $J/\psi$ sideband regions are
%shown as shadow histogram in plots, too. The backgrounds are well described by $J/\psi$ sideband events.
%%%%%%%%%%%%%%%%%%%%%
%\begin{figure*}[htbp]
%\begin{center}
%\begin{overpic}[width=7.0cm,height=5.0cm,angle=0]{graph/4230_fit_eta_mumu.eps}
%\put(25,60){\large\bf (a)}
%\end{overpic}
%\begin{overpic}[width=7.0cm,height=5.0cm,angle=0]{graph/4230_fit_eta_ee.eps}
%\put(25,60){\large\bf (b)}
%\end{overpic}
%\begin{overpic}[width=7.0cm,height=5.0cm,angle=0]{graph/4260_fit_eta_mumu.eps}
%\put(25,60){\large\bf (c)}
%\end{overpic}
%\begin{overpic}[width=7.0cm,height=5.0cm,angle=0]{graph/4260_fit_eta_ee.eps}
%\put(25,60){\large\bf (d)}
%\end{overpic}
%\end{center}
%\caption{The invariant mass distributions of two photons at $\sqrt{s}$ = 4.230 (a)(b)
%and 4.260 GeV (c)(d). The left two plots are for $J/\psi\to\mu^{+}\mu^{-}$ mode and
%right two for $J/\psi\to e^{+}e^{-}$ mode.
%Dots with error bars are data; the red solid curves for the total fitting results;
%the blue dotted curves for the background from fitting and the shadow histograms
%for the normalized $J/\psi$ sideband events.}
%\label{4230gg}
%\end{figure*}
%%%%%%%%%%%%%
%
%The process of $e^+e^-\to\pi^0 J/\psi$ also is searched for in $J/\psi\to\mu^+\mu^-$ decay
%mode by checking $M(\gamma \gamma)$ distribution around $\pi^0$ mass region.
%The searching is not performed with $J/\psi\to e^+e^-$ mode due to suffering from the huge
%background of radiative Bhabha events.
%Due to misidentification of $\pi^{\pm}$ to $\mu^{\pm}$, a peaking background from
%$e^{+}e^{-} \to \pi^{+}\pi^{-}\pi^{0}$ is observed around $\pi^0$ mass region on
%$M(\gamma \gamma)$ distribution for the events with two charged tracks invariant
%mass in the $J/\psi$ signal region, as well as in $J/\psi$ sideband region.
%A further requirement, at least one charged track has a muon counter hit depth larger
%than 30.0 cm, is applied to remove this background.
%Fig.~\ref{4230ggPi0} shows $M(\gamma \gamma)$ distributions around the $\pi^0$ mass
%region for event with two charged tracks invariant mass within the $J/\psi$ signal region
%at $\sqrt{s}$ = 4.230 and 4.260GeV, respectively.
%The corresponding plots for the events within the $J/\psi$ sideband regions are shown as
%shadow histograms after normalization, too. No significant $\pi^{0}$ signal is observed.
%
%%%%%%%%%%%%%%%%%%%%%
%\begin{figure*}[htbp]
%\begin{center}
%\begin{overpic}[width=7.0cm,height=5.0cm,angle=0]{graph/4230_fit_pi0_mumu.eps}
%\put(25,60){\large\bf (a)}
%\end{overpic}
%\begin{overpic}[width=7.0cm,height=5.0cm,angle=0]{graph/4260_fit_pi0_mumu.eps}
%\put(25,60){\large\bf (b)}
%\end{overpic}
%\end{center}
%\caption{The invariant mass distributions of two photons at $\sqrt{s}$ = 4.230 (a) and
%4.260 GeV (b) in $J/\psi\to\mu^{+}\mu^{-}$ mode.
%Dots with error bars are data and the shadow histograms
%are the normalized $J/\psi$ sideband events.}\label{4230ggPi0}
%\end{figure*}
%%%%%%%%%%%%%
%
%\section{\boldmath M($\gamma \gamma$) mass spectrum fits and cross section results}
%An unbinned maximum likelihood fit is performed on $M(\gamma\gamma)$ in $J/\psi\to e^{+}e^{-}$
%and $\mu^{+}\mu^{-}$ modes, respectively.
%The $\eta$ signal probability density function (PDF) is represented with a signal MC simulated
%shape convoluted with a Gaussian function, where the Gaussian function describes
%the resolution difference between data and MC simulation, and its parameters are floating in the fit.
%The background shape is described by a second-order Chebyshev polynomial function.
%The corresponding fit results at $\sqrt{s}$ = 4.230 and 4.260 GeV are shown in Fig.~\ref{4230gg},
%and the fit yields of $\eta$ signal are shown in Table~\ref{ResultsEta}.
%%%%%%%%%%%%%%%%%%%%
%\begin{table*}[htbp]
%  \footnotesize
%  \centering
%  \caption{The results on $e^{+}e^{-}\to\eta J/\psi$ at data samples with significant signal. The table shows the c.m energy $\sqrt{s}$, integrated luminosity $\mathcal{L}_{int}$, number of observed $\eta$ events $N^{obs}_{\eta}(\mu^{+}\mu^{-})$/$N^{obs}_{\eta}(e^{+}e^{-})$ from fitting, efficiency $\epsilon_{\mu}/\epsilon_{e}$, radiative correction factor $(1+\delta^{ISR})$, vacuum polarization factor $(1+\delta^{VP})$, Born cross section $\sigma^{B}(\mu^{+}\mu^{-})$/$\sigma^{B}(e^{+}e^{-})$  and Combined Born cross section $\sigma^{B}_{Com}$ (where the first uncertainties are statistical and the second are systematic).}
%  \begin{tabular}{cccccccccc}
%  \hline
%  \hline
%  $\sqrt{s}$(GeV) &$\mathcal{L}$(pb$^{-1})$ &$N^{obs}_{\eta}(\mu^{+}\mu^{-})$/$N^{obs}_{\eta}(e^{+}e^{-})$ &$\epsilon_{\mu}/\epsilon_{e}(\%)$ &$(1+\delta^{ISR})$ &$(1+\delta^{VP})$ &$\sigma^{B}(\mu^{+}\mu^{-})$(pb) &$\sigma^{B}(e^{+}e^{-})$(pb)  &$\sigma^{B}_{Com}$(pb)\\
%  $4.190$    &$43.1$    &$17.5\pm4.3$/$10.4\pm3.6$     &$35.6/24.8$      &$0.93$   &$1.056$   &$49.8\pm12.2\pm3.4$     &$42.5\pm14.7\pm2.4$       &$46.8\pm9.4\pm2.4$\\
%  $4.210$    &$54.6$    &$25.7\pm5.1$/$14.8\pm4.5$     &$31.7/22.0$      &$1.08$   &$1.057$   &$55.8\pm11.1\pm3.7$     &$46.3\pm14.1\pm3.9$       &$52.1\pm8.7\pm3.0$\\
%  $4.220$    &$54.1$    &$32.6\pm5.8$/$11.4\pm3.9$     &$29.6/20.4$      &$1.15$   &$1.057$   &$71.8\pm12.8\pm3.8$     &$36.5\pm12.5\pm1.7$       &$52.9\pm8.9\pm3.0$\\
%  $4.230$    &$1091.7$  &$394.3\pm20.9$/$274.9\pm20.1$ &$28.1/19.3$      &$1.18$   &$1.056$   &$44.3\pm2.3\pm1.7$      &$44.9\pm3.3\pm2.1$        &$44.5\pm1.9\pm1.6$\\
%  $4.245$    &$55.6$    &$9.3\pm3.3$/$9.7\pm3.6$       &$26.1/18.3$      &$1.26$   &$1.056$   &$20.7\pm7.3\pm2.1$      &$30.7\pm11.4\pm3.1$       &$23.6\pm6.1\pm1.9$\\
%  $4.260$    &$825.7$   &$94.4\pm10.5$/$75.9\pm11.9$   &$24.7/17.6$      &$1.28$   &$1.054$   &$14.7\pm1.6\pm0.6$      &$16.6\pm2.6\pm0.6$        &$15.3\pm1.3\pm0.5$\\
%  $4.360$    &$539.8$   &$19.8\pm5.3$/$23.9\pm7.7$     &$33.8/23.8$      &$0.95$   &$1.051$   &$4.7\pm1.2\pm0.4$       &$8.0\pm2.6\pm0.7$         &$5.3\pm1.1\pm0.5$\\
%  \hline
%  \hline
%    \end{tabular}
%    \label{ResultsEta}
%\end{table*}
%%%%%%%%%%%%
%
%The same event selection criteria are performed on the 13 data samples at different c.m. energy.
%Beside the data sample at $\sqrt{s}$ = 4.230 and 4.260 GeV, the $\eta$ signals are
%also observed clearly at $\sqrt{s}$ = 4.190, 4.210, 4.220, 4.245 and 4.360 GeV, respectively.
%The same fitting process is performed on these data samples, and the fit yields are
%shown in Table~\ref{ResultsEta}, too.
%Since there are not obvious $\eta$ signal observed in the other 6 c.m. energy points
%($\sqrt{s}$ = 3.810, 3.900, 4.090, 4.310, 4.390, 4.420 GeV), the upper limits at a
%90\% confidence level (C.L.) on the Born cross section are determined.
%We extract the upper limits with $J/\psi\to\mu^+\mu^-$ decay mode only.
%Since the statistics are very limited, the number of the observed events is obtained by
%counting the entries in $\eta$ signal region ($0.518<M(\gamma\gamma)<0.578$ GeV/c$^2$),
%and the number of background events in the signal region can be estimated with the number
%of events in $\eta$ sideband region or $J/\psi$ sideband region (with additional $\eta$
%signal mass window requirement) by assuming a flat distribution of background around signal regions,
%respectively.
%The $\eta$ sideband region is defined as $0.383<M(\gamma\gamma)<0.503$ GeV/c$^2$ and
%$0.593<M(\gamma\gamma)<0.713$ GeV/c$^2$, and $J/\psi$ sideband region is defined as
%$2.932<M(\mu^+\mu^-)<3.052$ GeV/c$^2$ and $3.142<M(\mu^+\mu^-)<3.262$ GeV/c$^2$, which are
%all four times on the size of its signal region.
%The number of the observed events in signal region ($N^{sg}_{\eta}$), as well as that in
%the $\eta$ sideband band regions ($N^{sb}_{\eta}$) and in the $J/\psi$ sideband bands
%region ($N^{sb}_{J/\psi}$) are all listed in Table~\ref{ResultsEtaUpper}.
%%%%%%%%%%%%%%%%%%%
%\begin{table*}[htbp]
%  \centering
%  \caption{The upper limits results on $e^{+}e^{-} \to \eta J/\psi$ in $\mu^{+}\mu^{-}$ mode. The table shows the c.m energy $\sqrt{s}$, integrated luminosity $\mathcal{L}_{int}$, number of observed $\eta$ events $N^{sg}_{\eta}$, number of background from $\eta$ sideband $N^{sb}_{\eta}$, number of background from $J/\psi$ sideband $N^{sb}_{J/\psi}$, efficiency $\epsilon$, upper limit of signal number with the consideration of selection efficiency $N^{up}_{\eta}/\epsilon$(at $90\%$ C.L.), radiative correction factor $(1+\delta^{ISR})$, vacuum polarization factor $(1+\delta^{VP})$, Born cross section $\sigma^{B}$ and upper limit on Born cross sections $\sigma^{B}_{up}$(at $90\%$ C.L.). The first uncertainties are statistical and the second systematic.}
%  \begin{tabular}{ccccccccccc}
%  \hline
%  \hline
%  $\sqrt{s}$(GeV) &$\mathcal{L}(pb^{-1})$ &$N^{sg}_{\eta}$ &$N^{sb}_{\eta}$ &$N^{sb}_{J/\psi}$  &$\epsilon (\%)$&$N^{up}_{\eta}/\epsilon$&$(1+\delta^{ISR})$ &$(1+\delta^{VP})$ &$\sigma^{B}$(pb) &$\sigma^{B}_{up} (pb)$\\
%  $3.810$     &$50.5$      &$5$         &$9$    &$11$   &$32.3$   &$<23.2$        &$1.24$       &$1.056$       &$5.5^{+7.1}_{-4.6}\pm0.6$                     &$<15.6$\\
%  $3.900$     &$52.8$      &$5$         &$8$    &$7$    &$39.6$   &$<19.6$        &$0.82$       &$1.049$       &$7.2^{+8.5}_{-5.5}\pm1.3$                     &$<19.3$\\
%  $4.090$     &$52.6$      &$7$         &$7$    &$5$    &$31.3$   &$<34.0$        &$1.00$       &$1.052$       &$13.1^{+9.7}_{-6.6}\pm2.0$                    &$<27.4$\\
%  $4.310$     &$44.9$      &$1$         &$4$    &$2$    &$25.9$   &$<10.5$        &$1.18$       &$1.053$       &$0.0^{+7.2}_{-2.8}\pm0.0$                     &$<8.6$\\
%  $4.390$     &$55.2$      &$5$         &$1$    &$4$    &$25.0$   &$<37.6$        &$1.32$       &$1.051$       &$10.7^{+7.7}_{-4.9}\pm0.6$                    &$<22.0$\\
%  $4.420$     &$44.7$      &$1$         &$3$    &$4$    &$20.3$   &$<14.6$        &$1.57$       &$1.053$       &$0.7^{+6.9}_{-2.6}\pm0.4$                     &$<8.8$\\
%  \hline
%  \hline
%    \end{tabular}
%    \label{ResultsEtaUpper}
%\end{table*}
%%%%%%%%%%%
%
%The Born-order cross section is determined from:
%\begin{center}
%\begin{equation}
%\sigma^{B}=\frac{N^{obs}}{\mathcal{L}_{int}\cdot (1+\delta^{ISR})\cdot (1+\delta^{VP})
%\cdot \epsilon \cdot \cal B }
%\label{eqcross}
%\end{equation}
%\end{center}
%where $N^{obs}$ is number of observed signal events,
%$\mathcal{L}_{int}$ is integrated luminosity,
%$(1+\delta^{ISR})$ is ISR correction factor which is obtained from QED calculation~\cite{QED}
%and taking the line shape of Born cross section measured by Belle experiment~\cite{belle} as input.
%$(1+\delta^{VP})$ is the vacuum polarization (VP) factor which is taken from QED calculation with an
%accuracy of $0.5\%$~\cite{VP},
%$\epsilon$ is selection efficiency,
%$\cal B$ is product branching ratio, $\BR(J/\psi \to \LL)\cdot \BR(\eta \to \gamma\gamma)$, cited from Particle Data Group (PDG)~\cite{PDG}.
%
%The final Born production cross sections of $e^+e^-\to \eta J/\psi$ at 7 c.m. energy points with significant signals are listed in Table~\ref{ResultsEta}.
%
%For the other 6 c.m. energy points in which the $\eta$ signal is not significant, we set
%upper limits at a 90\% C.L. on the Born cross section.
%The upper limit is calculated by a frequentist method with profile likelihood treatment
%of systematic uncertainties, which is implemented by a C++ class TROLKE in ROOT framework~\cite{TROLKE}.
%The number of observed signal events and estimated background events are assumed to follow Poisson distribution,
%and the efficiency are assumed to follow Gaussian distribution.
%Since the number of backgrounds can be estimated from $\eta$ sideband or $J/\psi$ sideband events, respectively, the
%conservative results of the upper limit on Born cross section are taken as the final results.
%The upper limits on the Born cross section at these 6 c.m. energy points are listed in
%Table ~\ref{ResultsEtaUpper}, respectively.
%
%Since there are not significant signals of $e^{+}e^{-} \to \pi^{0}J/\psi$ observed in all
%13 c.m. energy points, we set the upper limits at a 90\% C.L. on its Born cross section.
%The same frequentist method is implemented to extract the upper limits.
%Where, the number of observed events is obtained by counting the entries in $\pi^0$ signal
%region ($0.120<M(\gamma\gamma)<0.150$ GeV/c$^2$),
%and the number of background events in the signal region can be estimated with the number
%of events in the $\pi^0$ sideband region ($0.055<M(\gamma\gamma)<0.115$ GeV/c$^2$ and
%$0.155<M(\gamma\gamma)<0.215$ GeV/c$^2$) or $J/\psi$ sideband region
%($2.932<M(\mu^+\mu^-)<3.052$ GeV/c$^2$ and $3.142<M(\mu^+\mu^-)<3.262$ GeV/c$^2$)
%(with additional $\pi^0$ signal mass window requirement), and the conservative
%upper limits on the Born cross section are take as the final results.
%The number of observed events in the signal region ($N^{sg}$), in the $\pi^0$
%sideband region ($N^{sb}_{\pi^0}$) and in the $J/\psi$ sideband region ($N^{sb}_{J/\psi}$)
%as well as the upper limit at a 90\% C.L. on the Born cross section at 13 c.m. energy
%point are list in Table~\ref{Npi0Jpsi}.
%
%%%%%%%%%%%%%%%%%%%
%\begin{table*}[htbp]
%  \centering
%  \caption{The upper limits results on $e^{+}e^{-} \to \pi^{0} J/\psi$. The table shows the number of observed $\pi^{0}$ events $N^{sg}$, number of background by $\pi^{0}$ sideband events $N^{sb}_{\pi^{0}}$, number of background by $J/\psi$ sideband events $N^{sb}_{J/\psi}$, efficiency $\epsilon$, upper limit of signal number with the consideration of selection efficiency $N^{up}_{\pi^{0}}(\mu^{+}\mu^{-})/\epsilon$(at $90\%$ C.L.) and upper limit on Born cross sections $\sigma^{B}_{up}$(at $90\%$ C.L.).}
%  \begin{tabular}{ccccccccc}
%  \hline
%  \hline
%  $\sqrt{s}$(GeV) &$N^{sg}$  &$N^{sb}_{\pi^{0}}$  &$N^{sb}_{J/\psi}$ &$\epsilon (\%)$   &$N^{up}_{\pi^{0}}/\epsilon$  &$(1+\delta^{ISR})$ &$(1+\delta^{VP})$    &$\sigma^{B}_{up}$(pb)\\
%  $3.810$         &$1$       &$4$                 &$1$               &$16.9$            &$<20.2$                      &$1.24$             &$1.056$              &$<5.4$\\
%  $3.900$         &$0$       &$1$                 &$2$               &$28.3$            &$<6.2$                       &$0.82$             &$1.049$              &$<2.5$\\
%  $4.090$         &$0$       &$0$                 &$2$               &$20.1$            &$<9.9$                       &$1.00$             &$1.052$              &$<3.2$\\
%  $4.190$         &$0$       &$0$                 &$0$               &$23.1$            &$<8.6$                       &$0.93$             &$1.056$              &$<3.7$\\
%  $4.210$         &$1$       &$1$                 &$1$               &$21.1$            &$<16.2$                      &$1.08$             &$1.057$              &$<4.7$\\
%  $4.220$         &$0$       &$1$                 &$0$               &$20.1$            &$<9.9$                       &$1.15$             &$1.057$              &$<2.7$\\
%  $4.230$         &$4$       &$16$                &$13$              &$19.2$            &$<27.0$                      &$1.18$             &$1.056$              &$<0.4$\\
%  $4.245$         &$1$       &$1$                 &$2$               &$17.8$            &$<19.2$                      &$1.26$             &$1.056$              &$<4.8$\\
%  $4.260$         &$3$       &$8$                 &$10$              &$17.4$            &$<28.4$                      &$1.28$             &$1.054$              &$<0.5$\\
%  $4.310$         &$0$       &$0$                 &$0$               &$18.4$            &$<10.8$                      &$1.18$             &$1.053$              &$<3.5$\\
%  $4.360$         &$2$       &$3$                 &$4$               &$23.7$            &$<19.4$                      &$0.95$             &$1.051$              &$<0.6$\\
%  $4.390$         &$1$       &$1$                 &$1$               &$18.7$            &$<19.5$                      &$1.32$             &$1.051$              &$<4.6$\\
%  $4.420$         &$0$       &$2$                 &$2$               &$15.2$            &$<10.1$                      &$1.57$             &$1.053$              &$<2.4$\\
%  \hline
%  \hline
%    \end{tabular}
%    \label{Npi0Jpsi}
%\end{table*}
%%%%%%%%%%%%
%
%
%\section{Systematic Uncertainties}
%Several sources of systematic uncertainties are considered in the measurement of Born cross section.
%These include differences between data and MC simulation for the tracking algorithm, photon detection,
%kinematic fit, mass resolution, the fitting procedure, the hit depth in MUC, MC simulation of
%ISR correction factor and vacuum polarization factor, as well as branching fractions of
%intermediate states decays and luminosity measurement.
%
%(a) \emph{Tracking:} The uncertainty of tracking efficiency is investigated using a control sample
%$\psi(3686)\to\pi^{+}\pi^{-} J/\psi$ with subsequent decay of $J/\psi\to\ell^+\ell^-$.
%The difference in efficiency for lepton between data and MC is estimated to be 1\% per track,
%Therefore, 2\% uncertainty is taken as the systematic uncertainty for two leptons.
%
%(b) \emph{Photon detection efficiency:} The uncertainty due to photon detection and reconstruction is 1\% per photon~\cite{Gamma}.
%The value is determined from studies using clean control samples, such as $J/\psi\to\rho^{0}\pi^{0}$ and $e^{+}e^{-} \to \gamma\gamma$.
%Therefore, uncertainties of 2\% is taken for detection efficiency of two photons.
%
%(c) \emph{Kinematic fit:} The uncertainty associated with kinematic fit comes from the inconsistency on track helix
%parameters between data and MC simulation.
%Following the procedure described in Ref.~\cite{Helix}, we take the difference on the efficiencies
%between analyses with and without the helix parameters correction as the systematic uncertainty.
%
%(d) \emph{Mass window requirements:} A mass window requirement on $J/\psi$ mass will have systematic uncertainty on its efficiency.
%The $J/\psi$ signal of data at $\sqrt{s}$ = 4.230 GeV is fitted with a MC shape convoluted a Gaussian function, where the
%parameters of Guassian function are floating. The fits yield: mean value of Gaussian function is $2.8\pm0.5$ MeV in $\mu^{+}\mu^{-}$ mode
%and $2.4\pm1.1$ MeV in $e^{+}e^{-}$ mode; width of Gaussian function is $2.1\pm1.5$ MeV for $\mu^+\mu^-$ mode and $0.0\pm1.6$ MeV for $e^+e^-$ mode.
%To evaluated the systematic effects on the mass window requirement, the invariant mass of $\ell^+\ell^-$
%in MC samples are smeared with a Gaussian function, where the parameters of Gaussian function are
%obtained from the fit of $J/\psi$ signal of data.
%The difference on estimated efficiencies between the signal MC sample with and without mass
%resolution smearing is considered as the systematic uncertainty.
%%%%%%%%%%%%%%%%%%%
%\begin{table*}[htbp]
%  \centering
%  \caption{The summary of systematic uncertainties ($\%$) in the cross section measurement of $e^{+}e^{-} \to \eta J/\psi$
%  with significant signal in $\mu^{+}\mu^{-}$($e^{+}e^{-}$) mode. The common uncertainties (Luminosity, Tracking, Photon, Branching fraction and others) between two modes are shown together.}
%  \begin{tabular}{cccccccc}
%  \hline
%  \hline
%  Source/$\sqrt{s}$(GeV)    &4.190        &4.210        &4.220        &4.230        &4.245         &4.260         &4.360 \\
%  \hline
%  Luminosity                &1.0          &1.0          &1.0          &1.0          &1.0           &1.0           &1.0       \\
%  Tracking                  &2.0          &2.0          &2.0          &2.0          &2.0           &2.0           &2.0       \\
%  Photon                    &2.0          &2.0          &2.0          &2.0          &2.0           &2.0           &2.0       \\
%  Kinematic fit             &0.4 (0.4)    &0.4 (0.4)    &0.4 (0.3)    &0.4 (0.3)    &0.5 (0.5)     &0.4 (0.4)     &0.3 (0.4)  \\
%  Mass window               &0.2 (0.1)    &0.2 (0.1)    &0.2 (0.1)    &0.2 (0.1)    &0.2 (0.1)     &0.2 (0.1)     &0.2 (0.1)  \\
%  Fitting range             &0.0 (1.0)    &0.4 (0.7)    &0.3 (2.6)    &0.1 (2.2)    &0.0 (1.0)     &0.0 (0.6)     &8.6 (7.5)  \\
%  Signal shape              &0.3 (1.1)    &0.3 (1.1)    &0.3 (1.1)    &0.3 (1.1)    &0.3 (1.1)     &0.3 (1.1)     &0.3 (1.1)  \\
%  Background shape          &4.6 (0.1)    &3.9 (6.8)    &2.8 (0.0)    &0.0 (0.1)    &9.7 (9.3)     &0.2 (0.0)     &0.5 (0.4)  \\
%  ISR factor                &3.8 (4.3)    &4.0 (3.6)    &2.8 (1.3)    &1.9 (1.6)    &0.3 (0.5)     &2.6 (0.2)     &1.0 (2.0)  \\
%  Branching fraction        &1.2          &1.2          &1.2          &1.2          &1.2           &1.2           &1.2       \\
%  Others                    &1.0          &1.0          &1.0          &1.0          &1.0           &1.0           &1.0       \\
%  Sum                       &6.9 (5.7)    &6.6 (8.5)    &5.3 (4.6)    &3.9 (4.7)    &10.3 (10.0)   &4.3 (3.6)     &9.3 (8.6)  \\
%  \hline
%  \hline
%    \end{tabular}
%    \label{systematic1}
%\end{table*}
%%%%%%%%%%%
%
%(e) \emph{Fitting procedure:} For the 7 c.m. energy points with clearly observed $\eta$ signal, the fits on the two photons
%invariant mass $M(\gamma\gamma)$ are performed to extract the production of $e^+e^-\to \eta J/\psi$.
%The following three aspects are considered when evaluate the systematic uncertainty associated with
%the fit procedure.
%(1) {\it Fitting range:} In the nominal fit, the $M(\gamma\gamma)$ is fitted from a region from 0.2 to 0.9 GeV/c$^2$.
%  A alternative fit with different fitting range, 0.25 to 0.85 GeV/c$^2$, is performed, and results in the differences
%  on the production yield are treated as the systematic uncertainty from fitting range.
%(2) {\it Signal shape:} In the nominal fit, signal shape is described with MC shape convoluted
%   with a Gaussian function. A alternative fit with Crystal Ball function for the $\eta$ signal shape
%   is performed,  where the parameters of the Crystal Ball function in the different c.m. energy points
%   are fixed to those obtained from the fitting $\eta$ shape at $\sqrt{s}$ = 4.230 GeV.
%   The difference on the production yield are considered as the systematic uncertainty from the signal shape.
%(3) {\it Background shape:} In the nominal fit, background shapes are described as a second-order
%   polynomial function. The fitting with a third-order polynomial function for the background shape
%   is used to estimated the uncertainty from background shape.
%For the others 6 c.m. energy point without $\eta$ signal observed, the frequentist method is imposed to
%set the upper limits on the Born cross section, and number counting method is used to estimate
%the number of signal and background events. Two different sideband regions, $i.e.$ $\eta$ sideband region or
%$J/\psi$ sideband region, are used to estimated the uncertainty from the background estimation.
%The uncertainty has been considered in the upper limit setting by using the conservative upper limits
%as final results.
%
%(f) \emph{MUC cut:} In the searching for $e^{+}e^{-} \to \pi^{0} J/\psi$ study, a further requirement on hit depth in MUC
%for muon tracks is implemented to remove the background $e^{+}e^{-} \to \pi^+\pi^-\pi^{0}$.
%The corresponding systematic uncertainty, 1.2\%, which is estimated by study a control sample of
%$e^{+}e^{-} \to \pi^{+}\pi^{-} J/\psi$ with subsequent decay $J/\psi\to \mu^{+}\mu^{-}$, is taken
%as the systematic uncertainty.
%
%(g) \emph{ISR and VP factor:} The MC simulation uncertainty includes uncertainty associated with ISR correction factor and with
%vacuum polarization factor.
%The ISR correction factor uncertainty due to the line shape of cross section is estimated using
%the different line shapes from Belle experiment~\cite{belle} and the combination of Belle's and our own results.
%The uncertainty associated with vacuum polarization factor is $0.5\%$~\cite{VP}.
%
%(h) \emph{Luminosity:} The integrated luminosity of data samples used in this analysis are measured using large angle
%Bhabha events, and the corresponding uncertainties are estimated to be 1.0$\%$~\cite{Luminosity}.
%
%(i) \emph{Branching fraction:} The uncertainties due to branching fraction of $J/\psi \to \LL$, $\eta \to \gamma \gamma$
%and $\pi^{0} \to \gamma \gamma$ are taken from PDG ($2012$)~\cite{PDG}.
%
%(j) \emph{Other systematic uncertainties:} Other sources of systematic error, including trigger simulation, event start time determination,
%the total systematic error due to these sources is estimated to be less than $1.0\%$.
%
%Assuming all of the above systematic uncertainties, shown in Table~\ref{systematic1}, Table~\ref{systematic2} and Table~\ref{systematic3}, are
%independent, the total systematic uncertainties are obtained by adding the individual
%uncertainties in quadrature.
%
%%%%%%%%%%%%%%%%%%%
%\begin{table*}[htbp]
%  \centering
%  \caption{The summary of systematic uncertainties ($\%$) in the upper limit cross section measurement of $e^{+}e^{-} \to \eta J/\psi$ in $\mu^{+}\mu^{-}$ mode.}
%  \begin{tabular}{ccccccc}
%  \hline
%  \hline
%  Source/$\sqrt{s}$(GeV)    &3.810        &3.900        &4.090        &4.310        &4.390         &4.420  \\
%  \hline
%  Luminosity                &1.0          &1.0          &1.0          &1.0          &1.0           &1.0   \\
%  Tracking                  &2.0          &2.0          &2.0          &2.0          &2.0           &2.0   \\
%  Photon                    &2.0          &2.0          &2.0          &2.0          &2.0           &2.0   \\
%  Kinematic fit             &0.4          &0.4          &0.4          &0.4          &0.4           &0.4   \\
%  Mass window               &0.2          &0.2          &0.2          &0.2          &0.2           &0.2   \\
%  ISR factor                &0.2          &1.5          &0.1          &4.9          &2.3           &3.0   \\
%  Branching fraction        &1.2          &1.2          &1.2          &1.2          &1.2           &1.2   \\
%  Others                    &1.0          &1.0          &1.0          &1.0          &1.0           &1.0   \\
%  Sum                       &3.4          &3.7          &3.4          &6.0          &4.1           &4.5   \\
%  \hline
%  \hline
%    \end{tabular}
%    \label{systematic2}
%\end{table*}
%%%%%%%%%%%
%%%%%%%%%%%%%%%%%%%
%\begin{table*}[htbp]
%  \centering
%  \caption{The summary of systematic uncertainties ($\%$) in the cross section measurement of $e^{+}e^{-} \to \pi^{0} J/\psi$ in $\mu^{+}\mu^{-}$ mode.}
%  \begin{tabular}{cccccccccccccc}
%  \hline
%  \hline
%  Source/$\sqrt{s}$(GeV)  &3.810  &3.900  &4.090  &4.190  &4.210  &4.220  &4.230  &4.245  &4.260  &4.310  &4.360  &4.390  &4.420\\
%  Luminosity              &1.0    &1.0    &1.0    &1.0    &1.0    &1.0    &1.0    &1.0    &1.0    &1.0    &1.0    &1.0    &1.0\\
%  Tracking                &2.0    &2.0    &2.0    &2.0    &2.0    &2.0    &2.0    &2.0    &2.0    &2.0    &2.0    &2.0    &2.0\\
%  Photon                  &2.0    &2.0    &2.0    &2.0    &2.0    &2.0    &2.0    &2.0    &2.0    &2.0    &2.0    &2.0    &2.0\\
%  Kinematic fit           &0.4    &0.4    &0.4    &0.4    &0.4    &0.4    &0.4    &0.4    &0.4    &0.4    &0.4    &0.4    &0.4\\
%  Mass window             &0.2    &0.2    &0.2    &0.2    &0.2    &0.2    &0.2    &0.2    &0.2    &0.2    &0.2    &0.2    &0.2\\
%  MUC cut                 &1.2    &1.2    &1.2    &1.2    &1.2    &1.2    &1.2    &1.2    &1.2    &1.2    &1.2    &1.2    &1.2\\
%  ISR factor              &0.2    &1.5    &0.1    &3.8    &4.0    &2.8    &1.9    &0.3    &2.6    &4.9    &1.0    &2.3    &3.0\\
%  Branching fraction      &1.0    &1.0    &1.0    &1.0    &1.0    &1.0    &1.0    &1.0    &1.0    &1.0    &1.0    &1.0    &1.0\\
%  Others                  &1.0    &1.0    &1.0    &1.0    &1.0    &1.0    &1.0    &1.0    &1.0    &1.0    &1.0    &1.0    &1.0\\
%  Sum                     &3.6    &3.9    &3.6    &5.2    &5.4    &4.5    &4.0    &3.6    &4.4    &6.1    &3.7    &4.2    &4.7\\
%  \hline
%  \hline
%    \end{tabular}
%    \label{systematic3}
%\end{table*}
%%%%%%%%%%%
%
%For 7 c.m. energy points with significant signals, since the results from the two $J/\psi$ decay modes
%are consistent with each other within their uncertainties, the combined cross sections are calculated
%by considering the correlation of uncertainties between these two measurement~\cite{Combined}.
%
%\section{Summary and Discussion}
%In summary, using electron positron collision data samples collected with BESIII detector at
%13 c.m. energies from 3.810 to 4.420 GeV, we perform analysis of $e^{+}e^{-} \to \eta J/\psi$.
%The significant $\eta$ signals are observed at 7 c.m. energy points, and the corresponding
%Born cross sections are measured.
%The upper limits at a 90\% C.L. are set for the others 6 c.m. energies points without $\eta$ signal observed.
%Meanwhile, we search for the process of $e^{+}e^{-} \to \pi^0 J/\psi$, no significant signals are
%observed, and upper limits at a 90\% C.L. on the Born cross section are set for all 13 c.m. energy points.
%
%The measured Born cross section of $e^{+}e^{-} \to \eta J/\psi$ agree very well with the previous measurements~\cite{etaJpsi1,4040,belle}, but with much improved accuracy.
%Figure~\ref{BES_BELLE} (left) shows the comparison of Born cross section from Belle experiment and this analysis,
%a clear structure is observed around 4.220 GeV.
%The measured Born cross section also is compared to that of $e^{+}e^{-}
%\to \pi^{+}\pi^{-} J/\psi$ from Belle experiment in Fig.~\ref{BES_BELLE} (right), the two line shapes are not consistent
%well with each others, which indicates that the observed structure in $\eta J/\psi$ cross section
%does not arise from the $Y$(4260) state.
%The ratio of Born cross section at 4.260 GeV to that of 4.230 GeV,
%$R^{4.260}_{4.230}(e^+e^-\to\eta J/\psi)=\frac{\sigma^{4.260}(e^+e^-\to\eta J/\psi)}{\sigma^{4.230}(e^+e^-\to\eta J/\psi)}$,
%%%%%%%%%%%%%%%%%%%%%
%\begin{figure}[htbp]
%\begin{center}
%\begin{overpic}[width=7.0cm,height=5.0cm,angle=0]{graph/BES_BELLE_etaJpsi.eps}
%\put(25,60){\large\bf (a)}
%\end{overpic}
%\begin{overpic}[width=7.0cm,height=5.0cm,angle=0]{graph/BES_BELLE_PiPiJpsi.eps}
%\put(25,60){\large\bf (b)}
%\end{overpic}
%\end{center}
%\caption{The measured Born cross section of $e^{+}e^{-} \to \eta J/\psi$ (a) and $e^{+}e^{-} \to \pi^{+}\pi^{-} J/\psi$ (b). The black square dots with error are results from Belle, blue star dot is from BESIII(2012), and the red dots are from BESIII(2013).}\label{BES_BELLE}
%\end{figure}
%%%%%%%%%%%%%
%is calculated to be $0.34 \pm 0.04$ (the common systematic uncertainties cancel in the ratio calculation),
%which is found to agree very well with the ratio
%$R^{4.260}_{4.230}(e^+e^-\to\omega \chi_{c0})=\frac{\sigma^{4.260}(e^+e^-\to\omega \chi_{c0})}{\sigma^{4.230}(e^+e^-\to\omega \chi_{c0})}$ = $0.43 \pm 0.13$,
%for the process of $e^+e^-\to\omega \chi_{c0}$~\cite{OmegaChicj}. This may indicate that the production
%of $\eta J/\psi$ and $\omega \chi_{c0}$ are from same source probably. More data around these energy region
%may shed light to clarify this fact. The measured upper limits on Born cross section of $e^+e^-\to\pi^0 J/\psi$
%are within the the theoretical prediction~\cite{Theory2}.
%
%\acknowledgments
%The BESIII collaboration thanks the staff of BEPCII and the computing center for their strong support.
%This work is supported in part by the Ministry of Science and Technology of China under Contract No. 2009CB825200;
%Joint Funds of the National Natural Science Foundation of China under Contracts Nos. 11079008, 11179007, U1332201;
%National Natural Science Foundation of China (NSFC) under Contracts Nos.
%10625524, 10821063, 10825524, 10835001, 10935007, 11125525, 11235011, 11335008, 11275189, 11322544, 11375170;
%the Chinese Academy of Sciences (CAS) Large-Scale Scientific Facility Program;
%CAS under Contracts Nos. KJCX2-YW-N29, KJCX2-YW-N45;
%100 Talents Program of CAS;
%German Research Foundation DFG under Contract No. Collaborative Research Center CRC-1044;
%Istituto Nazionale di Fisica Nucleare, Italy;
%Ministry of Development of Turkey under Contract No. DPT2006K-120470;
%National Natural Science Foundation of China (NSFC) under Contract No. 11275189;
%U. S. Department of Energy under Contracts Nos. DE-FG02-04ER41291, DE-FG02-05ER41374, DE-FG02-94ER40823, DESC0010118;
%U.S. National Science Foundation;
%University of Groningen (RuG) and the Helmholtzzentrum fuer Schwerionenforschung GmbH (GSI), Darmstadt;
%WCU Program of National Research Foundation of Korea under Contract No. R32-2008-000-10155-0.
%
%
%
%\begin{thebibliography}{99}
%   %%Y(4260) and Y(4360) and Y(4660) states
%   \bibitem{Y(4260)1} B. Aubert {\it et al.}  (BABAR Collaboration), Phys.\ Rev.\ Lett.\ {\bf 95}, 142001 (2005);\\
%                      J. P. Lees {\it et al.} (BABAR Collaboration), Phys.\ Rev.\ D\     {\bf 86}, 051102(R) (2012).
%   \bibitem{Y(4360)1} B. Aubert {\it et al.}  (BABAR Collaboration), Phys.\ Rev.\ Lett.\ {\bf 98}, 212001 (2007);\\
%                      J. P. Lees {\it et al.} (BABAR Collaboration), Phys.\ Rev.\ D\     {\bf 89}, 111103(R) (2014).
%   \bibitem{Y(4260)2} C. Z. Yuan {\it et al.} (Belle Collaboration), Phys.\ Rev.\ Lett.\ {\bf 99}, 182004 (2007).
%   \bibitem{Y(4260)21}Z. Q. Liu  {\it et al.} (Belle Collaboration), Phys.\ Rev.\ Lett.\ {\bf 110}, 252002 (2013).
%   \bibitem{Y(4360)2} X. L. Wang {\it et al.} (Belle Collaboration), Phys.\ Rev.\ Lett.\ {\bf 99}, 142002 (2007);\\
%                      X. L. Wang for the Belle Collaboration, talk at the April APS meeting.
%   \bibitem{Y(4630)2} G. Pakhlova{\it et al.} (Belle Collaboration), Phys.\ Rev.\ Lett.\ {\bf 111}, 172001 (2008).
%   \bibitem{Y(4260)3} Q. He {\it et al.}      (CLEO Collaboration) , Phys.\ Rev.\ D\     {\bf 74}, 091104(R) (2006).
%
%   \bibitem{speVect}  T. Barnes, S. Godfrey and E. S. Swanson, Phys.\ Rev.\ D\  {\bf 72}, 054026 (2005).
%
%   %%Hadronic final states
%   \bibitem{DDstate}  B. Aubert {\it et al.}  (BABAR Collaboration), Phys.\ Rev.\ D\     {\bf 76}, 111105(R) (2007);\\
%                      B. Aubert {\it et al.}  (BABAR Collaboration), Phys.\ Rev.\ D\     {\bf 79}, 092001 (2009);\\
%                      G. Pakhlova {\it et al.} (Belle Collaboration), Phys.\ Rev.\ Lett.\ {\bf 98}, 092001 (2007);\\
%                      G. Pakhlova {\it et al.} (Belle Collaboration), Phys.\ Rev.\ D\    {\bf 77}, 011103(R) (2008).
%
%   \bibitem{hadfine}  T. E. Coan {\it et al.} (CLEO Collaboration) , Phys.\ Rev.\ Lett.\  {\bf 96}, 162003 (2006);\\
%                      G. Pakhlova {\it et al.} (Belle Collaboration), Phys.\ Rev.\ Lett.\ {\bf 100}, 062001 (2008);\\
%                      G. Pakhlova {\it et al.} (Belle Collaboration), Phys.\ Rev.\ Lett.\ {\bf 101}, 172001 (2008);\\
%                      G. Pakhlova {\it et al.} (Belle Collaboration), Phys.\ Rev.\ D\     {\bf 80}, 091101(R) (2009).
%   \bibitem{R-BESII}  J. Z. Bai {\it et al.} (BES Collaboration), Phys.\ Rev.\ Lett.\ {\bf 88}, 101802 (2002).
%
%   %Zc states
%   \bibitem{Zc(3900)1} M. Ablikim {\it et al.} (BESIII Collaboration), Phys.\ Rev.\ Lett.\ {\bf 110}, 252001 (2013).
%   \bibitem{Zc(3900)12}T. Xiao, S. Dobbs, A. Tomaradze, Kamal K. Seth, Phys.\ Rev.\ B.\ {\bf 727}, (2013) 366-370.
%   \bibitem{Zc(3885)1} M. Ablikim {\it et al.} (BESIII Collaboration), Phys.\ Rev.\ Lett.\ {\bf 112}, 022001 (2014).
%   \bibitem{Zc(4020)1} M. Ablikim {\it et al.} (BESIII Collaboration), Phys.\ Rev.\ Lett.\ {\bf 111}, 242001 (2013).
%   \bibitem{Zc(4025)1} M. Ablikim {\it et al.} (BESIII Collaboration), Phys.\ Rev.\ Lett.\ {\bf 112}, 132001 (2014).
%   \bibitem{Zc(3900)2} M. Ablikim {\it et al.} (BESIII Collaboration), BAM-00115;
%   \bibitem{Zc(4020)2} M. Ablikim {\it et al.} (BESIII Collaboration), BAM-00111.
%   \bibitem{OmegaChicj}M. Ablikim {\it et al.} (BESIII Collaboration), BAM-00121.
%
%   %hybrid charmonium
%   \bibitem{YNature1} F. E. Close and P.R. Page, Phys.\ Lett.\ B.\ {\bf 628}, 215 (2005). \\
%                      S. L. Zhu, Phys.\ Lett.\ B.\ {\bf 625}, 212 (2005). \\
%                      E. Kou and O. Pene, Phys.\ Lett.\ B.\ {\bf 631}, 164 (2005). \\
%                      X. Q. Luo and Y. Liu, hep-lat/0512044.
%
%   %tetraquarks
%   \bibitem{YNature5} D. Ebert, R. N. Faustov, and V. O. Galkin, Phys.\ Lett.\ B.\ {\bf 634}, 214 (2006). \\
%                      L. Maiani, V. Riquer, F. Piccinini and A. D. Polosa, Phys.\ Rev.\ D.\ {\bf 72}, 031502(R) (2005). \\
%                      T. W. Chiu and T. H. Hsieh(TWQCD Collaboration), heplat/0512029.
%
%   %hadronic molecules
%   \bibitem{YNature8} X. Liu, X. Q. Zeng and X. Q. Li, Phys.\ Rev.\ D.\ {\bf 72}, 054023(R) (2005). \\
%                      C. F. Qiao, Phys.\ Lett.\ B.\ {\bf 639}, 263 (2006). \\
%                      C. Z. Yuan, P. Wang and X. H. Mo, Phys.\ Lett.\ B.\ {\bf 634}, 399 (2006).
%
%   %eta Jpsi information
%   \bibitem{Theory1} C. F Qiao, http://lanl.arxiv.org/abs/1403.1918v1.
%   \bibitem{Theory2} Q. Wang, G. Li, X. H. Liu, and Q. Zhao, Phys.\ Rev.\ D.\ {\bf 84}, 014007(R) (2012).
%
%   %eta Jpsi information
%   %Observation of psi(4040) and psi(4160) decay into eta Jpsi
%   \bibitem{belle} X. L. Wang {\it et al.} (Belle Collaboration), Phys.\ Rev.\ D.\ {\bf 87}, 051101(R) (2013).
%   \bibitem{4040} M. Ablikim {\it et al.} (BESIII Collaboration), Phys.\ Rev.\ D.\ {\bf 86}, 071101(R) (2012).
%   \bibitem{etaJpsi1} T. E. Coan {\it et al.} (CLEO Collaboration), Phys.\ Rev.\ Lett.\ {\bf 96}, 162003 (2006).
%
%   %Design and construction of the BESIII detector.
%   \bibitem{NIM:DET} M. Ablikim {\it et al.} (BESIII Collaboration), Nucl.\ Instrum.\ Meth.\ A {\bf 614}, 345 (2010).
%
%   %"ConExc" new generator
%   %\bibitem{conexc} R. G. Ping, http://arxiv.org/abs/1309.3932.
%
%   %GEANT-4
%   \bibitem{GEANT4} S. Agostinelli {\it et al.}  (GEANT4 Collaboration), Nucl.\ Instrum.\ Meth.\ A {\bf 506}, 250 (2003).
%
%   %"KKMC" generator
%   \bibitem{KKMC} S. Jadach, B. F. L. Ward and Z. Was, Comp.\ Phys.\ Commu.\ 130, 260 (2000); Phys.\ Rev.\ {bf D63}, 113009 (2001)
%   \bibitem{Evtgen:1} K. T. Chao {\it et al.}, Modern\ Physics\ A, 24 N0.1 supp. (2009).
%   \bibitem{Evtgen:2} R. G. Ping, Chin.\ Phys.\ C 32, 599 (2008).
%
%   \bibitem{LUNDCHARM}
%   J. C. Chen at el. Phys. Rev. D62, 034003(2000).
%
%
%   %(1+delta) Radiative correction factors
%   \bibitem{QED} E. A. Kuraev and V. S. Fadin, Yad. Fiz.41, 733-742(1985).
%
%   %Vacuum polarization
%   \bibitem{VP} http://www-com.physik.hu-berlin.de/~fjeger.
%
%   %PDG
%   \bibitem{PDG}
%   J. Beringer {\it et al.} (Particle Data Group), Phys.\ Rev.\ D.\ {\bf 86}, 010001 (2012).
%
%   %TROLKE
%   \bibitem{TROLKE} Rolke {\em et al.}  Nucl.\ Instrum.\ Meth.\ A {\bf 551}, 439 (2005).
%
%   \bibitem{Gamma} M. Ablikim {\em et al.} (BESIII Collaboration), Phys. Rev. D {\bf 81}, 052005 (2010).
%
%   \bibitem{Helix} M. Ablikim {\em et al.} (BESIII Collaboration), Phys. Rev. D {\bf 87}, 012002 (2013).
%
%   %Integrated Luminosity measurement
%   \bibitem{Luminosity} M. Ablikim {\it et al.} (BESIII Collaboration), BAM-00110.
%
%   %combined%
%   \bibitem{Combined} M. Ablikim {\em et al.} (BESIII Collaboration), Phys. Rev. D {\bf 89}, 074030 (2014).
%
%\end{thebibliography}

\end{document}
